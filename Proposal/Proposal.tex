\documentclass{article}
\addtolength{\textwidth}{1.5in}
\addtolength{\textheight}{2.5in}
\addtolength{\hoffset}{-1in}
\addtolength{\voffset}{-1in}
\addtolength{\topmargin}{-10pt}
\title{Math 199: \LaTeX\mbox{} Tutorial}
\author{Tom Schenk Jr.}
\begin{document}
\maketitle
\section{Focus of the Course}

The focus of the course is to create a \LaTeX\mbox{} tutorial for Drake University students. \LaTeX\mbox{} is an open--source typesetting program for technical documents. The program is particularly adept to easily creating mathematical symbols and bibliographies. It is often considered a standard for authors in mathematics, physics, computer sciences, and economics. This course will create an online tutorial with an emphasis on the aspects of \LaTeX\mbox{} likely to be used throughout undergraduate study. The tutorial will be experimented with the Department of Mathematics and Computer Science FYS course. It will undoubtly be a benefit for students throughout their undergraduate coursework, for those going to graduate school, and those creating documents in the private sector.

\section{Course Outline}

The tutorial will primarily consist of PDF slides (created by myself), supplemental documents (also written by myself), and 3$^{rd}$--party documents already freely available. An emphasis will be placed on the slides and the 3$^{rd}$--party documents freely available online or exclusively to Drake University students (e.g. Cowles Library, ebrary, etc.) This tutorial will be distinctly different from most other online tutorials by specifically focusing on the needs to undergraduate students.

The tentative outline of the course is divided into two unequal sections. The first three to four meetings (the interval of meetings is to be determined) will introduce the fundamental commands for \LaTeX. After this section, an astute participant should be able to use \LaTeX\mbox{} for seventy percent of papers she will be assigned. The section section consists of six to eight meetings. These meetings will explore more advanced aspects of \Latex\mbox{} such as graphics and ``floats''. We will also attempt to integrate \LaTeX\mbox{} with other software a student may encounter, such as Mathematica. A student participating in both sections would be able to complete any University assignment in \LaTeX. Below is a table outlining the current tentative outline.
\\
\\
\begin{tabular}{|l|l|l|}
\hline
Meeting \#1 & Introduction to \LaTeX & Why \LaTeX?; Basic Structure; Needed Software \\
\hline
Meeting \#2 & Structure & Document Types; Generating Documents \\
\hline
Meeting \#3 & Math \& Tables & Math; Tables \\
\hline
Meeting \#4 & Automatic Indexing & Bibliographies; Index of Tables; Indexes \\
\hline
\multicolumn{3}{|c|}{\textit{At this time students should be able to prepare intermediate \LaTeX\mbox{} documents}} \\
\hline
Meeting \#5 & Advanced Mathematics & Mathematica; HTML; \LaTeX\mbox{}, IDE \\
\hline
Meeting \#6 & Advancing Formatting & Floats; Graphics \\
\hline
Meeting \#7 & Advancing Formatting II & Advanced Floats; Advanced Graphics \\
\hline
Meeting \#8 & Multilingual & Babel \\
\hline
Meeting \#9 & Multiple .tex Files & Style Files \\
\hline 
\end{tabular}

\section{Course Goals}

Throughout the project I will be emphasizing two goals: to continue to use \LaTeX\mbox{} in their coursework and to contribute to the \LaTeX\mbox{} project. The former can be satisfied by emphasizing the benefits of \LaTeX. The latter will be achieved by making the source material available to be edited and extended. That is, documents I write for this tutorial will be made available in PDF for viewing and .tex for editing. My hope is that making the source code available will let a future student modify the tutorial as the \LaTeX\mbox{} program changes.

\end{document}
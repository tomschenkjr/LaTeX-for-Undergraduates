%PLEASE READ=============================
%This document is copyrighted by Tom Schenk Jr and Drake University (2005). It may be modified and re-compiled for personal use only. A re-compiled version may not be distributed without permission of the copyrighters.
%Authors modifying this document for LaTeX for Undergraduates may append their name(s) in the ``\author'' command. They may not delete the names of previous authors.
\documentclass{article}
%================================Packages
\usepackage{graphicx}
\usepackage{verbatim}   % useful for program listings
\usepackage{color}      % use if color is used in text
\usepackage{subfigure}  % use for side-by-side figures

\usepackage{amsmath}    % need for subequations
%================================Document Info.
\title{\textsc{\LaTeX\ for Undergraduates\\
			Introduction} \\
			Lecture Notes}
\author{Tom Schenk Jr.}
\date{\textit{Version of \today}}
%================================Body
\begin{document}

\maketitle

\section*{Example Document}

This output is an excerpt from a sample document written by Harvey Gould. To view the source code, return to the website and open the \texttt{.txt} file.

\section{Output}

\begin{center}
Introduction to \LaTeX \\   % \\ = new line
Harvey Gould \\
January 16, 2005
\end{center}

\section{Introduction}
\TeX\ looks more difficult than it is. It is
almost as easy as $\pi$. See how easy it is to make special
symbols such as $\alpha$,
$\beta$, $\gamma$,
$\delta$, $\sin x$, $\hbar$, $\lambda$, $\ldots$ We also can make
subscripts
$A_{x}$, $A_{xy}$ and superscripts, $e^x$, $e^{x^2}$, and
$e^{a^b}$. We will use \LaTeX, which is based on \TeX\ and has
many higher-level commands (macros) for formatting, making
tables, etc. More information can be found in Ref.~\cite{latex}.

We just made a new paragraph. Extra lines and spaces make no
difference. Note that all formulas are enclosed by
\$ and occur in \textit{math mode}.

The default font is Computer Modern. It includes \textit{italics}
or {\it italics}, \textbf{boldface} or {\bf boldface},
\textsl{slanted} or {\sl slanted}, and \texttt{monospaced} or {\tt
monospaced} (typewriter) fonts.

\section{Equations}
Let us see how easy it is to write equations.
\begin{equation}
\Delta =\sum_{i=1}^N w_i (x_i - \bar{x})^2 .
\end{equation}
It is a good idea to number equations, but we can have a
equation without a number by writing
\begin{equation}
P(x) = \frac{x - a}{b - a} , \nonumber
\end{equation}
and
\begin{equation}
g = \frac{1}{2} \sqrt{2\pi} . \nonumber
\end{equation}
Note the different ways of writing a ratio.

We can give an equation a label so that we can refer to it later.
\begin{equation}
\label{eq:ising}
E = -J \sum_{i=1}^N s_i s_{i+1} ,
\end{equation}
Equation~(\ref{eq:ising}) expresses the energy of a configuration
of spins in the Ising model.\footnote{It is necessary to process (typeset) a
file twice to get the counters correct.}

We can define our own macros to save typing. For example, suppose
that we introduce the macros:
\begin{verbatim}
 \newcommand{\lb}{{\langle}}
 \newcommand{\rb}{{\rangle}}
\end{verbatim}
\newcommand{\lb}{{\langle}}
\newcommand{\rb}{{\rangle}}
Then we can write the average value of $x$ as
\begin{verbatim}
\begin{equation}
\lb x \rb = 3
\end{equation}
\end{verbatim}
The result is
\begin{equation}
\lb x \rb = 3 .
\end{equation}

Examples of more complicated equations:
\begin{equation}
I = \! \int_{-\infty}^\infty f(x)\,dx \label{eq:fine}.
\end{equation}
We can do some fine tuning by adding small amounts of horizontal
spacing:
\begin{verbatim}
 \, small space       \! negative space
\end{verbatim}
as is done in Eq.~(\ref{eq:fine}).

We also can align several equations:
\begin{align}
a & = b \\
c &= d ,
\end{align}
or number them as subequations:
\begin{subequations}
\begin{align}
a & = b \\
c &= d .
\end{align}
\end{subequations}

We can also have different cases:
\begin{equation}
\label{eq:mdiv}
m(T) =
\begin{cases}
0 & \text{$T > T_c$} \\
\bigl(1 - [\sinh 2 \beta J]^{-4} \bigr)^{\! 1/8} & \text{$T < T_c$}
\end{cases}
\end{equation}
write matrices
\begin{eqnarray}
\textbf{T} &=&
\begin{pmatrix}
T_{++} \hfill & T_{+-} \\
T_{-+} & T_{--} \hfill 
\end{pmatrix} , \nonumber \\
& =&
\begin{pmatrix}
e^{\beta (J + B)} \hfill & e^{-\beta J} \hfill \\
e^{-\beta J} \hfill & e^{\beta (J - B)} \hfill
\end{pmatrix}.
\end{eqnarray}
and 
\newcommand{\rv}{\textbf{r}}
\begin{equation}
\sum_i \vec A \cdot \vec B = -P \! \int \! \rv \cdot
\hat{\mathbf{n}}\, dA = P \! \int \! {\vec \nabla} \cdot \rv\, dV.
\end{equation}

\section{Tables}
Tables are a little more difficult until you get the knack. TeX
automatically calculates the width of the columns.

\begin{table}[h]
\begin{center}
\begin{tabular}{|l|l|r|l|}
\hline
lattice & $d$ & $q$ & $T_{\rm mf}/T_c$ \\
\hline
square & 2 & 4 & 1.763 \\
\hline
triangular & 2 & 6 & 1.648 \\
\hline
diamond & 3 & 4 & 1.479 \\
\hline
simple cubic & 3 & 6 & 1.330 \\
\hline
bcc & 3 & 8 & 1.260 \\
\hline
fcc & 3 & 12 & 1.225 \\
\hline
\end{tabular}
\caption{\label{tab:5/tc}Comparison of the mean-field predictions
for the critical temperature of the Ising model with exact results
and the best known estimates for different spatial dimensions $d$
and lattice symmetries.}
\end{center}
\end{table}

\section{Lists}

Some example of formatted lists include the
following:

\begin{enumerate}

\item bread

\item cheese

\end{enumerate}

\begin{itemize}

\item Tom

\item Dick

\end{itemize}

\begin{thebibliography}{5}

\bibitem{latex}Helmut Kopka and Patrick W. Daly, \textsl{A Guide to
\LaTeX: Document Preparation for Beginners and Advanced Users},
third edition, Addison-Wesley (1999).

\bibitem{website}Some useful links are
given at {\tt {<}http://sip.clarku.edu/tutorials/TeX/{>}}.

\end{thebibliography}

\end{document}


\documentclass[pdf]{prosper}
%================================Packages
\usepackage{hyperref}
\usepackage{mflogo}
\usepackage{fancybox}
\usepackage{multicol}
%================================Document Info./Slide 1
\title{Preamble}
\subtitle{Generate Documents -- Windows}
\author{Tom Schenk Jr.}
\institution{Drake University}
\email{tls007@drake.edu}
%================================Document Body
\begin{document}
\maketitle
%================================Slide 2
\begin{slide}{This Presentation}
	\begin{itemize}
		\item Generating documents from source code.
		\item Using \TeX nicCenter.
		\item Outputting to DVI in DOS.
		\item Outputting to PostScript in DOS.
		\item Outputting to PDF in DOS.
	\end{itemize}
\end{slide}
%================================Slide 3
\begin{slide}{Generating Documents}
	\begin{itemize}
		\item Documents can be generated in two ways: from a program and from DOS.
		\item Generating documents in a \LaTeX\ program is very easy.
		\item Generating documents in DOS is a bit more complicated, but more powerful.
	\end{itemize}
\end{slide}
%================================Slide 4
\begin{slide}{Using TeXnicCenter}
	\begin{itemize}
		\item A drop--down menu can be used to choose the output format (DVI, PDF, and PS for PostScript).
		\item Use ``Build Output'' (F7) when working with a project and ``Build Current File'' (Ctrl + F5) when working with a single file.
		\item ``View Output'' (F5) will display the compiled document.
		\item The Build menu offers more build options that will be used later in the tutorial.
	\end{itemize}
\end{slide}
%================================Slide 5
\begin{slide}{Using DOS - Review}
	\begin{itemize}
		\item If \LaTeX\ is installed, but \TeX nicCenter is not, you must use DOS.
		\item For Windows 2000 and Windows XP, access DOS through the Start Menu $\rightarrow$ Run, then type ``cmd.''
			\begin{itemize}
				\item Do not use ``command,'' this does not show appropriate file names.
			\end{itemize}
		\item Alternatively, you may access Run with Windows Key + R.
	\end{itemize}
\end{slide}
%================================Slide 6
\begin{slide}{Using DOS - Review (con't)}
	\begin{itemize}
		\item \texttt{dir} will display a list of files and folders in that directory.
		\item \texttt{cd \textit{folder}} will open the subfolder.
		\item \texttt{cd..} will return you up a level, \texttt{cd$\backslash$} will return to the root level (e.g. C:$\backslash$).
	\end{itemize}
\end{slide}
%================================Slide 7
\begin{slide}{DOS: Outputting to DVI}
	\begin{itemize}
		\item Find the directory where the source file (.tex file) is located.
		\item \texttt{latex \textit{filename}} will output the document to DVI.
			\begin{itemize}
				\item Omit the .tex extension.
			\end{itemize}
		\item It will notify the user if there are warnings or errors.
			\begin{itemize}
				\item Note the error so you may return to the file and correct it.
			\end{itemize}
	\end{itemize}
\end{slide}
%================================Slide 8
\begin{slide}{DOS: Outputting to PostScript}
	\begin{itemize}
		\item After creating a DVI, \texttt{dvips} transforms it to PostScript.
		\item Options are invoked with \texttt{dvips -\textit{option filename}}
			\begin{itemize}
				\item \texttt{-t a4} will output to the A4 format.
				\item \texttt{-A} will print only odd pages, \texttt{-B} will print even only.
				\item A list of options can be accessed with \texttt{dvips}.
			\end{itemize}
	\end{itemize}
\end{slide}
%================================Slide 9
\begin{slide}{DOS: Outputting to PDF}
	\begin{itemize}
		\item There are two ways of creating PDFs in DOS: pdf\LaTeX\ and ps2pdf.
		\item \texttt{pdflatex \textit{filename}} can transform the .tex file directly to PDF.
		\item The downside is that it will not include bookmarks and if you use encapsulated postscripts for images, compiling may fail.
		\item In other words, pdf\LaTeX\ works for basic documents.
		\item The other option is \texttt{ps2pdf \textit{filename}.ps}, which translates the PostScript (.ps) to PDF.
		\item You must include the .ps after the filename.
	\end{itemize}
\end{slide}
%================================Slide 10	
\begin{slide}{Review}
	\begin{itemize}
		\item \TeX nicCenter is the easiest way to compile documents.
		\item DVI, PostScript, and PDFs can be created in DOS.
		\item \texttt{latex \textit{filename}} will output to DVI.
		\item \texttt{dvips \textit{filename}} generates a PostScript document.
		\item \texttt{pdflatex \textit{filename}} or \texttt{ps2pdf \textit{filename}.ps} will make a PDF document.
	\end{itemize}
\end{slide}

\end{document}
\documentclass[pdf]{prosper}
%================================Packages
\usepackage{hyperref}
\usepackage{mflogo}
\usepackage{fancybox}
\usepackage{multicol}
%================================Document Info./Slide 1
\title{Body}
\subtitle{Spacing, Paragraphs, and Sectioning}
\author{Tom Schenk Jr.}
\institution{Drake University}
\email{tls007@drake.edu}
%================================Document Body
\begin{document}
\maketitle
%================================Slide 2
\begin{slide}{This Presentation}
	\begin{itemize}
		\item Spacing
		\item Creating linebreaks.
		\item Creating paragraphs
		\item Creating chapters and sections.
	\end{itemize}
\end{slide}
%================================Slide 2
\begin{slide}{Spacing}
	\begin{itemize}
		\item \LaTeX\ feeds the input document line--by--line and assembles it according to your commands and document class.
		\item When it sees multiple spaces, it treats it as one ordinary space.
		\item Trying to create a certain ``look'' by using a lot of spaces is difficult in \LaTeX.
		\item This is because \LaTeX\ is trying to organize the document using a logical and consistent fashion.
		\item Nevertheless, we can use the \texttt{$\backslash$hspace\{\textit{length}\}} to create a long space.	\\
		Ex: \texttt{Here is a \texttt{$\backslash$hspace\{\textit{1in} long space.\}}} \\
		Here is a \hspace{1in} long space. \\
	\end{itemize}
\end{slide}
%================================Slide 2
\begin{slide}{Creating Linebreaks}
	\begin{itemize}
		\item There are two ways of creating linebreaks:
			\begin{itemize}
				\item Typing a return at the end of the line (Enter key).
				\item Using two backslashes ($\backslash$$\backslash$) at the end of the line.
			\end{itemize}
		\item Typically, using the enter key is adequate.
		\item $\backslash$$\backslash$ is sometimes needed while making tables.
		\item If the output file doesn't have a linebreak where it should, then use $\backslash$$\backslash$.
	\end{itemize}
\end{slide}
%================================Slide 2
\begin{slide}{Creating Paragraphs}
	\begin{itemize}
		\item Using two line breaks---either two returns or two sets of $\backslash$$\backslash$ will create a new paragraph.
		\item \LaTeX\ automatically indents for a normal document (\texttt{article},\texttt{book}, etc. classes).
		\item The author may use commands to force indentation or to tell \LaTeX\ not to.
		\item \texttt{$\backslash$noindent} forces \LaTeX\ not to indent the new paragraph.
		\item \texttt{$\backslash$indent} forces \LaTeX\ to create an indentation.
		\item Use either command before starting a new paragraph---creating an indent is by default.
	\end{itemize}
\end{slide}
%================================Slide 2
\begin{slide}{Creating Paragraphs (con't)}
	\begin{itemize}
		\item An author may not want to indent or may want to change how big the default indentation is.
		\item We can use \texttt{$\backslash$setlength\{$\backslash$parindent\}} in the preamble to change the indentation.
	\end{itemize}
			\texttt{$\backslash$documentclass[pdf]\{article\}} \\
			\vdots
			\texttt{$\backslash$setlength\{$\backslash$parindent\}\{\#pt\}} \\
			\texttt{$\backslash$begin\{document\}} \\
	\begin{itemize}
		\item Insert a number to represent the number of pixel spaces it will indent---0 will have no indent.
	\end{itemize}
\end{slide}
%================================Slide 2
\begin{slide}{Creating Paragraphs (con't)}
	\begin{itemize}
		\item Likewise we may want to adjust the space between paragraphs (from where one ends to where the next one begins).
		\item We can use \texttt{$\backslash$setlength\{$\backslash$parskip\}} in the preamble.
			\texttt{$\backslash$documentclass[pdf]\{article\}} \\
			\vdots
			\texttt{$\backslash$setlength\{$\backslash$parskip\}\{\#pt plus \#pt minus \#pt\}} \\
			\texttt{$\backslash$begin\{document\}} \\
		\item The first \# tells \LaTeX your preferred length, but can vary by ``plus \#'' or ``minus \#'' if necessary.
	\end{itemize}
\end{slide}
%================================Slide 2
\begin{slide}{Creating Paragraphs (con't)}
	\begin{itemize}
		\item More importantly, we can change the spacing between lines (i.e. double spacing).
		\item You can use another form of a \texttt{$\backslash$setlength}, but the \texttt{setspace} package is tremendously powerful.
			\texttt{$\backslash$documentclass[pdf]\{article\}} \\
			\texttt{$\backslash$usepackage\{setspace\}} \\
			\vdots
			\texttt{$\backslash$begin\{document\}} \\
			\texttt{$\backslash$doublespacing} \\
			\texttt{Double---spaced text.} \\
			\texttt{$\backslash$singlespacing} \\
			\texttt{Now we're back to single---spaced.} \\
			\texttt{$\backslash$onehalfspacing} \\
			\texttt{There is also one and a half spacing.} \\
	\end{itemize}
\end{slide}
%================================Slide 2
\begin{slide}{Creating Paragraphs (con't)}
	\begin{itemize}
		\item When using the \texttt{setspace} package, everything after \texttt{$\backslash$doublespacing}, \texttt{$\backslash$singlespacing}, \texttt{$\backslash$onehalfspacing}.
		\item You can put \texttt{$\backslash$doublespacing}, \texttt{$\backslash$singlespacing}, or \texttt{$\backslash$onehalfspacing} in the preamble so the entire document is the same.
		\item However, you may not want to double space certain parts of your document (i.e. abstract), so you may want to place it later in the document.
	\end{itemize}
\end{slide}
%================================Slide 2
\begin{slide}{Chapters and Sections}
	\begin{itemize}
		\item One of the most powerful features of \LaTeX\ is sectioning.
		\item \LaTeX will automatically number chapters, sections, and subsections.
		\item These will automatically appear in our table of contents if we choose to have one.
		\item The most basic command is \texttt{$\backslash$section\{\textit{Section Name}\}}, with the \textit{section name} and number appearing at the top of that section.
			\texttt{$\backslash$begin\{document\}} \\
			\texttt{$\backslash$section\{Introduction\}} \\
			\texttt{Introductory portion of our paper\ldots} \\
			\texttt{$\backslash$section\{Experiment\}} \\
			\texttt{Experimental portion---\LaTeX\ will automatically number these sections.} \\
	\end{itemize}
\end{slide}
%================================Slide 2
\begin{slide}{Chapters and Sections (con't)}
	\begin{itemize}
		\item \texttt{$\backslash$section} works for all classes.
		\item \texttt{$\backslash$chapter} can only be used in the \texttt{report} or \texttt{book} class.
		\item \texttt{$\backslash$chapter\{\textit{Chapter Name}\}} will create a chapter with the \textit{chapter name} and number at the top.
		\item You can use the \texttt{chapter} and \texttt{section} commands in the same document.
	\end{itemize}
\end{slide}
%================================Slide 2
\begin{slide}{Chapters and Sections (con't)}
	\begin{itemize}
		\item You may also want to create subsections under other sections. This can be done with the \texttt{$\backslash$subsection\{\textit{Subsection Name}\}} command.
		\item \LaTeX\ also uses \texttt{$\backslash$paragraph\{\textit{Paragraph Name}\}} and \texttt{$\backslash$subparagraph\{\textit{Subparagraph Name}\}}, although they are infrequently used.
		\item If you have chapters and want to separate the book into parts, you may use the \texttt{$\backslash$part\{\textit{Part Name}\}} command, which won't interfere with chapter numbers.
	\end{itemize}
\end{slide}
%================================Slide 2
\begin{slide}{Chapters and Sections (con't)}
	\begin{itemize}
		\item Sometimes we may not want to include chapter, section, subsection, etc. numbers in our paper.
		\item \texttt{$\backslash$section*\{\textit{Section Name}\}} and \texttt{$\backslash$chapter*\{\textit{Chapter Name}\}} will put the title above the section or chapter, but without the autonumbering.
		\item You may use the * to also stop other sectioning commands from numbering.
	\end{itemize}
\end{slide}
%================================Slide 2
\begin{slide}{Chapters and Sections (con't)}
	\begin{itemize}
		\item If we're using a table of contents, some of our chapters or sections may have long names that we don't want to have in our table of contents.
		\item We can use brackets to make a name that appears in the body of our paper and another name in the table of contents.
		\item \texttt{$\backslash$section[\textit{Table of Content's Name}]\{\textit{The Longer Name for the Actual Paper}\}} command will parse out two different section names.
		\item The same command can be used with parts, chapters, subsections, etc.
		\item It can also use the asterisk to exclude autonumbering.
	\end{itemize}
\end{slide}
%================================Slide 2
\begin{slide}{Chapters and Sections (con't)}
	\section{Section}
	Some text\ldots
	\subsection{Subsection}
	More text\ldots
	\section*{Numberless Section}
	Even more text\ldots
	\section{Section}
	And finally\ldots
\end{slide}
%================================Slide 2
\begin{slide}{Spacing and Sectioning Review}
	\begin{itemize}
		\item Linebreaks can be created using the Return key or $\backslash$$\backslash$.
		\item A pair of linebreaks will create a new paragraph.
		\item There are a handful of ways to change the line spacing in the document.
		\item Chapters and sections are autonumbering and ties in with the table of contents.
	\end{itemize}
\end{slide}
\end{document}
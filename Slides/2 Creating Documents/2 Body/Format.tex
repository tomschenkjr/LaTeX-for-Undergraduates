\documentclass[pdf]{prosper}
%================================Packages
\usepackage{hyperref}
\usepackage{mflogo}
\usepackage{fancybox}
\usepackage{multicol}
%================================Document Info./Slide 1
\title{Body}
\subtitle{Formatting}
\author{Tom Schenk Jr.}
\institution{Drake University}
\email{tls007@drake.edu}
%================================Document Body
\begin{document}
\maketitle
%================================Slide 2
\begin{slide}{This Presentation}
	\begin{itemize}
		\item Special characters
		\item Bolding, Italics, and Emphasizing
		\item Font sizes
		\item Justification
		\item Bulleted lists
		\item Enumerated lists
	\end{itemize}
\end{slide}
%================================Slide 3
\begin{slide}{Special Characters}
	\begin{itemize}
		\item As we've seen so far, several characters are used in commands
		\item To produce these characters in the body, we need to use a modified version. \\
			\begin{tabular}{l l l l l l l l l}
				\texttt{$\backslash$\$} & \texttt{$\backslash$\%} & \texttt{$\backslash$\&} & \texttt{$\backslash$\#} & \texttt{$\backslash$\{} & \texttt{$\backslash$\}} & \texttt{$\backslash$\_} & \texttt{$\backslash$\^{}\{\}} & \texttt{$\backslash$\~{}\{\}} \\
				\$ & \% & \& & \# & \{ & \} & \_ & \^{} & \~{} \\
			\end{tabular}
		\item To produce $\backslash$, use \texttt{\$$\backslash$backslash\$}.
		\item \^{} and \~{} overflow their space so they need extra brackets to create extra space.
			\begin{itemize}
				\item This is what \^ and \~ characters look like without adding the brackets.
			\end{itemize}
	\end{itemize}
\end{slide}
%================================Slide 2
\begin{slide}{Logo's}
	\begin{itemize}
		\item \LaTeX has some built in logo's.
			\begin{center}
				\begin{tabular}{|l|l|}
					\hline
					Command & Logo \\
					\hline
					$\backslash$LaTeX$\backslash$ & \LaTeX\ \\
					\hline
					$\backslash$LaTeX2e$\backslash$ & \LaTeX\ \\
					\hline
					$\backslash$TeX$\backslash$ & \TeX\ \\
					\hline
				\end{tabular}
			\end{center}
		\item These command tend to overflow their box, so the extra slash is to give it more space.
	\end{itemize}
\end{slide}
%================================Slide 2
\begin{slide}{Bolding, Italicizing, and Emphasizing}
	\begin{itemize}
		\item \LaTeX uses commands to format text.
			\begin{center}
				\begin{tabular}{r l}
					\hline
					Command & Output \\
					\hline
					$\backslash$textbf\{\textit{text}\} & \textbf{bold} \\
					$\backslash$textit\{\textit{text}\} & \textit{italic} \\
					$\backslash$texttt\{\textit{text}\} & \texttt{typewriter} \\
					$\backslash$textsl\{\textit{text}\} & \textsl{slanted} \\
					$\backslash$emph\{\textit{text}\} & \emph{emphasized} \\
					$\backslash$textsf\{\textit{text}\} & \textsf{sans serif} \\
					$\backslash$textsc\{\textit{text}\} & \textsc{small caps} \\
				\end{tabular}
			\end{center}
	\end{itemize}
\end{slide}
%================================Slide 2
\begin{slide}{Font Sizes}
	\begin{itemize}
		\item The normal font size for our document is defined by the \textit{document class} (default is 10pt).
		\item We can change the font size periodically throughout the document.
		\item Typically the author should rely on section commands (e.g. \texttt{$\backslash$section}) to make headers.
		\item Likewise, we should rely on the \texttt{$\backslash$maketitle} command.
	\end{itemize}
\end{slide}
%================================Slide 2
\begin{slide}{Font Sizes}
		\begin{center}
				\begin{tabular}{l l}
					\hline
					Command & Output \\
					\hline
					$\backslash$tiny & \tiny tiny font \\
					$\backslash$scriptsize & \scriptsize script size font \\
					$\backslash$footnotesize & \footnotesize footnote size font \\
					$\backslash$small & \small small font \\
					$\backslash$normalsize & \normalsize normal font \\
					$\backslash$large & \large large font \\
					$\backslash$Large & \large larger font \\
					$\backslash$LARGE & \LARGE even larger font \\
					$\backslash$huge & \huge huge font \\
					$\backslash$huge & \Huge largest font \\
				\end{tabular}
			\end{center}
\end{slide}
%================================Slide 2
\begin{slide}{Justification}
	\begin{itemize}
		\item To justify text (left, center, right) we ``encapsulate'' the text we want to justify in an environment.
			\begin{itemize}
				\item Environments are important in \LaTeX\ and we'll explore other environments in Lecture 3.1.
			\end{itemize}
		\item We can justify the text to the left (flushleft), center (center), or to the right (flushright).
		\item To begin a justification we use \texttt{$\backslash$begin\{\textit{justification}\}}.
		\item To end a justification we use \texttt{$\backslash$end\{\textit{justification}\}}.
		\item The \texttt{$\backslash$begin} and \texttt{$\backslash$end} commands are used for many scenarios.
	\end{itemize}
\end{slide}
%================================Slide 2
\begin{slide}{Justification Example}
This input\ldots \\
\\
			\texttt{$\backslash$begin\{document\}} \\
			\texttt{$\backslash$Text is normally justified to the left.} \\
			\texttt{$\backslash$begin\{center\}} \\
			\texttt{Here is some text we've centered.} \\
			\texttt{The same can be justified to the left,} \\
			\texttt{and to the right.} \\
			\texttt{$\backslash$end\{center\}} \\
			\texttt{Back to normal justification.} \\
\end{slide}
%================================Slide 2
\begin{slide}{Justification Example (con't)}
produces this output\dots \\
\\
	Text is normally justified to the left. \\
	\begin{center}
		Here is some text we've centered. \\
		The same can be justified to the left, \\
		and to the right.
	\end{center}
	Back to normal justification.
\end{slide}
%================================Slide 2
\begin{slide}{Bulleted Lists}
	\begin{itemize}
		\item It is also important to make bulleted or enumerated lists.
		\item Again, we will use an environment to make lists.
		\item To begin a bulleted list, we use \texttt{$\backslash$begin\{itemize\}}.
		\item To end the list, we use \texttt{$\backslash$end\{itemize\}}.
		\item To add a new bullet, use \texttt{$\backslash$item}.
		\item You can use a dash(-) instead of a bullet with \texttt{$\backslash$item[-]}.
		\item To make a sublist, begin another bulleted list.
	\end{itemize}
\end{slide}
%================================Slide 2
\begin{slide}{Bulleted List Example}
This input\ldots \\
\\
			\texttt{$\backslash$begin\{document\}} \\
			\texttt{$\backslash$begin\{itemize\}} \\
			\texttt{$\backslash$item The first point} \\
			\texttt{$\backslash$begin\{itemize\}} \\
			\texttt{$\backslash$item A subpoint to the first} \\
			\texttt{$\backslash$end\{itemize\}} \\
			\texttt{$\backslash$item[-] then the second with a slash} \\
			\texttt{$\backslash$end\{itemize} \\
			\texttt{Back to normal text.} \\
\end{slide}
%================================Slide 2
\begin{slide}{Bulleted List Example (con't)}
produces this output\ldots \\
\\
\\
			\begin{itemize}
				\item The first point
					\begin{itemize}
						\item A subpoint to the first
					\end{itemize}
				\item[-] then the second with a slash
			\end{itemize}
			Back to normal text.
\end{slide}
%================================Slide 2
\begin{slide}{Enumerated Lists}
	\begin{itemize}
		\item Making a numbered list is very similar.
		\item To begin a enumerated list, we use \texttt{$\backslash$begin\{enumerate\}}.
		\item To end the list, we use \texttt{$\backslash$end\{enumberate\}}.
		\item To add an item to the list, use \texttt{$\backslash$item}.
		\item To begin a sublist, start another enumerated list.
		\item \LaTeX\ allows authors to mix bulleted and enumerated lists.
	\end{itemize}
\end{slide}
%================================Slide 2
\begin{slide}{Enumerated List Example}
This input\ldots \\
			\texttt{$\backslash$begin\{document\}} \\
			\texttt{$\backslash$begin\{enumerated\}} \\
			\texttt{$\backslash$item Our first priority} \\
			\texttt{$\backslash$begin\{enumerated\}} \\
			\texttt{$\backslash$item A subitem to the enumerated list} \\
			\texttt{$\backslash$end\{enumerated\}} \\
			\texttt{$\backslash$begin\{itemize\}} \\
			\texttt{$\backslash$item With a bulleted list} \\
			\texttt{$\backslash$end\{itemize\}} \\			
			\texttt{$\backslash$item Our second priority} \\
			\texttt{$\backslash$end\{enumerated\}} \\
\end{slide}
%================================Slide 2
\begin{slide}{Enumerated List Example (con't)}
	\begin{enumerate}
		\item Our first priority
			\begin{enumerate}
				\item A subitem to the enumerated list
			\end{enumerate}
			\begin{itemize}
				\item With a bulleted list
			\end{itemize}
		\item Our second priority
	\end{enumerate}
\end{slide}
%================================Slide 2
\begin{slide}{Formating Review}
	\begin{itemize}
		\item A handful of characters are used in commands, so for them to actually appear in the text we need special commands.
		\item We can format text by using a list of commands (on slide 4) to format a section of text.
			\begin{itemize}
				\item \texttt{$\backslash$textit\{\textit{text}\}}, \texttt{$\backslash$textbf\{\textit{text}\}} will italicize and bold text, respectively.
			\end{itemize}
		\item We can make lists using $\backslash$begin\{itemize\} or $\backslash$begin\{enumerate\} to begin a list, \texttt{$\backslash$item} to insert a new item, and $\backslash$end\{itemize\} or $\backslash$begin\{enumerate\} to end the list.
	\end{itemize}
\end{slide}
%================================Slide 2
\begin{slide}{Creating Documents Review}
	\begin{itemize}
		\item A document is made up of two parts: the preamble and body.
			\begin{itemize}
				\item The preamble sets how the document will look.
				\item The body contains the actual text.
			\end{itemize}
		\item 
		\item NEXT: Intermediate documents.

\end{document}
\documentclass[pdf]{prosper}
%================================Packages
\usepackage{hyperref}
\usepackage{mflogo}
\usepackage{fancybox}
\usepackage{multicol}
%================================Document Info./Slide 1
\title{Preamble}
\subtitle{Classes and Packages}
\author{Tom Schenk Jr.}
\institution{Drake University}
\email{tls007@drake.edu}
%================================Document Body
\begin{document}
\maketitle
%================================Slide 2
\begin{slide}{This Presentation}
	\begin{itemize}
		\item Brief Recap of \LaTeX\mbox{} files.
		\item How to start and end a document.
		\item Paragraphs and linebreaks.
		\item Chapters, sections, and subsections.
		\item Title, author, and date headers.
	\end{itemize}
\end{slide}
%================================Slide 3
\begin{slide}{Body of a Document}
	\begin{itemize}
		\item Body of the document consists of the material that will be seen in the final print.
		\item The layout of the text and features we use depend on what we put in the preamble.
		\item This presentation briefly introduced formatting, but will be covered more extensively in the next presentation.
	\end{itemize}
\end{slide}
%================================Slide 4
\begin{slide}{Paragraphs}
	\begin{itemize}
		\item Hitting return after a line will create a new line.
		\item Alternatively, the author can use \texttt{$\backslash$$\backslash$} to create line breaks.
		\item Two returns or two \texttt{$\backslash$$\backslash$} will create a new paragraph.
		\item \LaTeX\mbox{} will automatically create indentations unless you instruct it not to.
	\end{itemize}
\end{slide}
%================================Slide 5
\begin{slide}{Paragraphs (con't)}
	\begin{itemize}
		\item If you want a new paragraph without indentation, use the  (block style).
		\item \texttt{$\backslash$indent} can be used to create an indent when a new paragraph does not contain one.
		\item Refer to the supplemental document for examples of document breaks.
	\end{itemize}
\end{slide}
%================================Slide 6
\begin{slide}{Chapters and Sections}
	\begin{itemize}
		\item Occasionally we want to break out document into into sections.
		\item Books are necessarily divided into chapters, and perhaps sections.
		\item When using the \texttt{Book} class, we may use \texttt{$\backslash$Chapter\{\textit{name}\}} command to create a new chapter with \textit{Name} at the beginning of the chapter.
		\item We can force chapters to start on the right--hand or left--hand page using either \texttt{[openright]} or \texttt{[openleft]} option. By default, chapters will start on the right--hand side for the \texttt{book} class.		
		\item \texttt{$\backslash$part\{\textit{name}\} will insert a part without interfering with chapter numbers.
	\end{itemize}
\end{slide}
%================================Slide 7
\begin{slide}{Chapters and Sections}
	\begin{itemize}
		\item \texttt{$\backslash$section\{\textit{name}\}} will create a numbered section for our document with \textit{Name} at the top of the section.
		\item \texttt{$\backslash$subsection\{\textit{name}\}} will create a numbered subsection for our document with \textit{Name} at the top of the subsection.
		\item We can keep adding subsections under subsections with more ``sub'' prefixes---i.e. \texttt{$\backslash$subsubsubsection}.
		\item Section and subsection commands will work for \texttt{article} or \texttt{book} classes. However, \texttt{$\backslash$chapter\{\textit{name}\}} will only work for the \texttt{book} class.
	\end{itemize}
\end{slide}
%================================Slide 8
\begin{slide}{Chapters and Sections (con't)}
	\begin{itemize}
		\item Chapters, section, subsections, etc., are also automatically numbered
\section{Section}
\subsection{Subsection}
\subsubsection{Subsubsection}
\section{Another Section}
		\item If a \texttt{$\backslash$section} command is inserted before others, then the sections will be renumbered.
		\item \texttt{$\backslash$section*\{\textit{name}\} will create a section header without numbers.
	\end{itemize}
\end{slide}
%================================Slide 9
\begin{slide}{Chapters and Sections (con't)}
	\begin{itemize}
		\item Parts, chapters, sections, and subsection numbers and name will be included in the table of contents if we choose to have one.
		\item Sometimes our sections will have a lengthy name that we do not want to include in the table of contents.
		\item Thus, we are able to give our sections a name that will appear in the full text and brief name for the table of contents.
		\begin{center}
			\texttt{$\backslash$section[\itshape{Table of Contents Title}]\{\textit{Full Name for the Main Text}\}
		\end{center}
	\end{itemize}
\end{slide}
%================================Slide 10
\begin{slide}{Title, Author, and Date}
	\begin{itemize}
		\item For homework, teachers often require a title, name, and date.
		\item By adding some lines in the preamble, we can make \LaTeX insert these lines where we request.
		\item Remember the preamble is from \texttt{$\backslash$documentclass\{\textit{class}\} to \texttt{$\backslash$begin\{document\}.
	\end{itemize}
\end{slide}
%================================Slide 10
\begin{slide}{Title, Author, and Date (con't)}
	\begin{itemize} 
		\item After we declare our document class and packages, we can create title and author fields.
			\begin{tabular}{l}
				\texttt{$\backslash$documentclass[\textit{options}]\{\textit{class}\}}
				\texttt{$\backslash$usepackage[\textit{options}]\{\textit{package}\}}
				\texttt{$\backslash$author\{\textit{Title of Paper}\}
				\texttt{$\backslash$author\{\textit{Author's Name}\}
			\end{tabular}
		\item After \texttt{$\backslash$begin\{document\}, we can insert \texttt{$\backslash$maketitle} where we want the title, author, and date (date is automatically stamped to when we compile the document).
	\end{itemize}
\end{slide}
%================================Slide 10
\begin{slide}{Title, Author, and Date (con't)}
	\begin{itemize} 
		\item Here is what our document looks like if we want the title, author, and date at the top of our document.
			\begin{tabular}{l}
				\texttt{$\backslash$documentclass[\textit{options}]\{\textit{class}\}}
				\texttt{$\backslash$usepackage[\textit{options}]\{\textit{package}\}}
				\texttt{$\backslash$author\{\textit{Title of Paper}\}
				\texttt{$\backslash$author\{\textit{Author's Name}\}
				\texttt{$\backslash$begin\{document\}
				\texttt{$\backslash$maketitle}
			\end{tabular}
	\end{itemize}
\end{slide}				
%================================Slide 11	
\begin{slide}{Title, Author, and Date (con't)}
	\begin{itemize}
		\item Sometimes we might want to insert a specific date instead of the date the document was made.
		\item To add a specific date, we add \texttt{$\backslash$date\{\textit{date}\} to the preamble:
		\begin{tabular}{l}
				\texttt{$\backslash$documentclass[\textit{options}]\{\textit{class}\}}
				\texttt{$\backslash$usepackage[\textit{options}]\{\textit{package}\}}
				\texttt{$\backslash$author\{\textit{Title of Paper}\}
				\texttt{$\backslash$author\{\textit{Author's Name}\}
				\texttt{$\backslash$date\{\textit{date}\}
				\texttt{$\backslash$begin\{document\}
				\texttt{$\backslash$maketitle}
			\end{tabular}
	\end{itemize}
\end{slide}
%================================Slide 11	
\begin{slide}{Review}
	\begin{itemize}
		\item We can use returns or \texttt{$\backslash$$\backslash$} to create linebreaks and paragraphs.
		\item \texttt{$\backslash$noindent} and \texttt{$\backslash$indent} can prevent or create indentations.
		\item \texttt{$\backslash$chapter\{\textit{chapter name}\} creates chapters in the \texttt{book} and \texttt{report} classes.
		\item \texttt{$\backslash$section\{\textit{section name}\} and \texttt{$\backslash$subsection\{\textit{subsection name}\} can create section and subsections in \texttt{article}, \texttt{book}, and \texttt{report} documents.
		\item We can easily create title and author headings which we can simply call with the \texttt{$\backslash$maketitle} command.
		\item NEXT: We will go over basic formatting typically used in papers.
\end{document}
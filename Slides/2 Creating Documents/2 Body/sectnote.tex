\documentclass{article}
%================================Packages
\usepackage{makeidx}						%For making the index
\usepackage{setspace}
%================================Document Info.
\title{\textsc{\LaTeX\ for Undergraduates\\
			Linebreaks, Paragraphs, and Sections} \\
			Lecture Notes}
\author{Tom Schenk Jr.}
\date{\textit{Version of \today}}
%================================Body
\begin{document}

\maketitle

\section{Motivation}

This lecture is the first on the body of a \LaTeX\ document. The material in the body is what appears in the documents output. This lecture reviews some basic commands: creating linebreaks and paragraphs. It will also introduce sectioning commands. Sectioning is a new concept for WYSIWYG users since it utilizes one of the powerful features of \LaTeX\ : autonumbering.

\section{Body}

The border between the preamble and body is the \texttt{$\backslash$begin\{document\}} command. As it implies, this command tells \LaTeX\ when to start the actual document. The information that appears after this command is considered the body. Information that appears before this command is the preamble.

\subsection{Linebreaks}

Linebreaking, or creating a new line, might seem like a trite topic, but it still worth mentioning. Normally, hitting return is adequate for making a new line; however, sometimes it is necessary to use a command to create a linebreak---especially a linebreak after a series of commands. The author may use \texttt{$\backslash$$\backslash$} to create a new linebreak.

\subsection{Paragraphs}

A new paragraph is created after two consecutive linebreaks. By default, \LaTeX\ will indent the newly created paragraphs, but this can be stopped with the \texttt{$\backslash$noindent} command before the paragraph. Likewise, if the \textit{document class} do not indent by default, then the author may use \texttt{$\backslash$indent} to create an indent.

The author may also change the size of the indentation by using the \texttt{$\backslash$setlength} in the \textit{preamble} (recall that the preamble is everything before \texttt{$\backslash$begin\{document\}}). \texttt{$\backslash$setlength\{$\backslash$\{parindent\}\{\textit{\#}pt\}} where \# is a number that sets the indent length. This command will apply the indentation parameter to the entire document.

Normally, professors require that the document is double spaced. By default, \LaTeX\ outputs in single line spacing. The \texttt{setspace} package lets the author change the spacing. \texttt{$\backslash$singlespacing}, \texttt{$\backslash$onehalfspacing}, and \texttt{$\backslash$doublespacing} will change the documents spacing after each corresponding command. That is, a portion of the document can be single spaced, while other parts are double or one half spaced.

\subsection{Sections}

You've been seen several examples of sections in this and previous lecture notes. Sections are similar to chapters, except on a smaller scale. They are a useful way of presenting a structured argument in a document. Sections are created with \texttt{$\backslash$section\{\textit{Name}\}}, where \textit{Name} is the name of the section. Likewise, you can create subsections using \texttt{$\backslash$subsection\{\textit{Name}\}}. You can create subsub\ldots section by simply adding an additional ``sub.'' This section's name is ``Body'' and the subsection is ``Sections.''

As you may have noticed, sections and subsections are numbered. \LaTeX\ automatically numbers each section, so if you were rearrange the ordering of a section, it will be renumbered. Likewise, the section names and numbers will be included in the table of contents, if you choose to have one (table of contents covered in Lecture 3.2.1).

Chapters are created using \texttt{$\backslash$chapter\{\textit{Name}\}} command. However, chapters can only be used with the \texttt{book} or \texttt{report} class. \texttt{$\backslash$part\{\textit{Name}\}} will separate the document into parts without interfering with the chapter numbers.

If you do not want to number any of the sectioning commands (e.g. section, chapter, or part), then include an asterisk in the command: \texttt{$\backslash$section*\{\textit{Name}\}}, \texttt{$\backslash$chapter*\{\textit{Name}\}}, etc.

Overall, I recommend that you make use of the sectioning commands. It will make the organization of your paper clearer and easier to read.

\section{Conclusion}

We've covered a bit of ground in the last few tutorials. Below is a sample code incorporating some things learned in the last few lectures:

\subsection{Sample Code}

\begin{verbatim}
\documentclass{article}
\usepackage{setspace}
\begin{document}
\doublespacing
\section*{Introduction}
Here is an example of double--spaced text explained in this tutorial. 
\LaTeX\ normally outputs with single--spaced text, but the 
$\backslash$doublespacing allows double spacing. It is important, however, 
to include the \texttt{setspace} package in the \textit{preamble}. This is 
an excellent example of the interaction between the preamble and the body.

\onehalfspacing
Now we turn to a one--half spacing example. This sort of spacing isn't used 
as much, but is a good substitute for double--spacing. Notice that we can 
combine several different spacing styles within one document. However, this 
sort of dichotomous typing is rarely used.
\end{document}
\end{verbatim}

\subsection{Sample Output}
\section*{Introduction}
\doublespacing
Here is an example of double--spaced text explained in this tutorial. \LaTeX\ normally outputs with single--spaced text, but the $\backslash$doublespacing allows double spacing. It is important, however, to include the \texttt{setspace} package in the \textit{preamble}. This is an excellent example of the interaction between the preamble and the body.

\onehalfspacing
Now we turn to a one--half spacing example. This sort of spacing isn't used as much, but is a good substitute for double--spacing. Notice that we can combine several different spacing styles within one document. However, this sort of dichotomous typing is rarely used.


\end{document}
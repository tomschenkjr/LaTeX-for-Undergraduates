\documentclass[pdf]{prosper}
%================================Packages
\usepackage{hyperref}
\usepackage{mflogo}
\usepackage{fancybox}
\usepackage{multicol}
%================================Theorems
\newtheorem{thm}{Theorem}
\newtheorem{lem}{Lemma}
%================================Document Info./Slide 1
\title{Advanced Mathematical Documents}
\subtitle{Theorem Environment}
\author{Tom Schenk Jr.}
\institution{Drake University}
\email{tls007@drake.edu}
%================================Document Body
\begin{document}
\maketitle
%================================Slide 2
\begin{slide}{This Presentation}
	\begin{itemize}
		\item The package.
		\item Creating theorems.
	\end{itemize}
\end{slide}
%================================Slide 3
\begin{slide}{Preamble Commands}
	\begin{itemize}
		\item Since there are theorems, lemmas, conjectures, etc., the author is allowed to create different style.
		\item This is done in the preamble with:
			\begin{center}
				\texttt{$\backslash$newtheorem\{\textit{abbrev.}\}\{\textit{Name}\}}
			\end{center}
		\item The \textit{abbrev.} is how the command will be started, \textit{Name} is what will be displayed in the document.
		\item Ex: \texttt{$\backslash$newtheorem\{\textit{Thm}\}\{Theorem\}}
	\end{itemize}
\end{slide}
%================================Slide 4
\begin{slide}{Creating a Theorem}
	\begin{itemize}
		\item In the body, the theorem environment starts and ends with:
			\begin{center}
				\texttt{$\backslash$begin\{\textit{abbrev.}\}} \\
				\texttt{$\backslash$end\{\textit{abbrev.}\}}
			\end{center}
		\item Using the previous example, the command would look like this:
			\begin{center}
				\texttt{$\backslash$begin\{thm\}} \\
				\texttt{$\backslash$end\{thm\}}
			\end{center}		
	\end{itemize}
\end{slide}
%================================Slide 5
\begin{slide}{Theorem Example}
\begin{verbatim}
\begin{thm} Let lim$_{x \rightarrow a}$ 
$f(x) = L.$ Then there exists 
$\delta > 0$ such that if 
$0 < \left|x-a\right| < \delta,$ 
then $\left|f(x\right| < 1 + \left|L\right|$.
\end{thm}
\end{verbatim}
\smallskip
Output:
\smallskip
\begin{thm} Let $\lim_{x \rightarrow a}$ $f(x) = L.$ Then there exists $\delta > 0$ such that if $0 < \left|x-a\right| < \delta,$ then $\left|f(x)\right| < 1 + \left|L\right|$.
\end{thm}
\end{slide}
%================================Slide 6
\begin{slide}{Conclusion}
	\begin{itemize}
		\item The author can add or adjust preamble commands to make theorems appear as Lemmas, Conjectures, or whatever is needed.
		\item It is important to remember to use the \$ symbol in the theorem environment, that is, the theorem environment is \emph{not} a math environment.
	\end{itemize}
\end{slide}

\end{document}
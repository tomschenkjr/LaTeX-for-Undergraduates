%This document is provided to the general audience to (1) give readers a template to learn and (2) to let other students update and improve the material when needed. Anyone is allowed to modify this document provided they do not remove these lines or do not remove my name from this document. If you do make improvements to this tutorial, I welcome those authors to provide their name alongside the materials they improve.
%�2005 Tom Schenk Jr.
%================================Slides
\documentclass[pdf]{prosper}
%================================Packages
\usepackage{hyperref}
\usepackage{mflogo}
\usepackage{fancybox}
%================================Document Info./Slide 1
\title{Preamble}
\subtitle{Classes and Packages}
\author{Tom Schenk Jr.}
\institution{Drake University}
\email{tls007@drake.edu}
%================================Document Body
\begin{document}
\maketitle
%================================Slide 2
\begin{slide}{This Presentation}
	\begin{itemize} 
		\item Structure of \LaTeX Documents
		\item Preamble
		\item Classes
		\item Class Options
		\item Packages
		\item Package Options
	\end{itemize}
\end{slide}
%================================Slide 3
\begin{slide}{Structure of \LaTeX document}
	\begin{itemize}
		\item Let's focus on the source file for documents (.tex).
		\item Every file has the same structure.
		\item A document is composed of a preamble and a body.
		\item The preamble tells \LaTeX how the document is going to look (e.g. font size, margins) and the body is the actual text.
		\item We will concentrate on the preamble in these next two presentations.
		\item This presentation will look at Classes and Packages.
	\end{itemize}
\end{slide}
%================================Slide 4
\begin{slide}{Structure of \LaTeX document}
	\begin{quote}
		\texttt{$\backslash$documentclass[pdf]\{article\}} \\
		\texttt{$\backslash$usepackage\{hhref\}} \\
		\texttt{$\backslash$title\{Exploitation and Unfreedom}
		\texttt{$\backslash$author\{Trin Turner\}} \\
		\texttt{$\backslash$institution\{Drake University\}} \\
		\texttt{$\backslash$begin\{document\}} \\
		\texttt{$\backslash$maketitle} \\
		\texttt{Some part of the body...} \\
		\texttt{$\backslash$end{document}} \\
	\end{quote}
	\begin{itemize}
		\item Above is a generic layout of \LaTeX\mbox{} commands for a very basic document.
	\end{itemize}
\end{slide}
%================================Slide 5
\begin{slide}{Preamble}
	\begin{itemize}
		\item Preamble contains information that will be used by \LaTeX when it generates a document.
		\item The first two lines of a document are \texttt{$\backslash$documentclass} and \texttt{$\backslash$usepackage} commands.
		\item \texttt{$\backslash$documentclass} tells \LaTeX what kind of document you are creating.
		\item \texttt{$\backslash$usepackage} will load features you want to enable for the document.
	\end{itemize}
\end{slide}
%================================Slide 6
\begin{slide}{Classes}
	\begin{itemize}
		\item Classes inform \LaTeX what kind of document you are creating.
		\item This is important because the layout of one type of document (e.g. article) is different from another (e.g. book)
		\item The Class command also lets you change the default options for a document (e.g. font size, paper format)
		\item The first line of every document should be the Class command\footnote{Actually, the author may insert comments that will not be read by \LaTeX before the Class command. Beginning a line with \% will tell \LaTeX to ignore everything on that line.}:
			\begin{center}			
				\texttt{$\backslash$documentclass[\textit{options}]\{\textit{class}\}}
			\end{center}
	\end{itemize}
\end{slide}
%================================Slide 7
\begin{slide}{Class}
	\begin{itemize}
		\item The Class command consists of two operators, \textit{class} and \textit{options}.
		\item \textit{Class} must be included because it specifically tells \LaTeX the document type.
			\begin{itemize}
				\item The next slide lists common classes.
			\end{itemize{
		\item \itextit{Options} lets the author choose her own default parameters for the document (e.g. font size)
		\item Using the \texttt{[\textit{options}]} command is optional, but the author \emph{must} define the class:
			\begin{center}
				\begin{tabular}{l l}
					Acceptable: & \texttt{$\backslash$documentclass\{article\}} \\
					Acceptable: & \texttt{$\backslash$documentclass[12pt]\{article\}} \\
					Not Acceptable: & \texttt{$\backslash$documentclass[12pt]}
				\end{tabular}
			\end{center}
		\item When the author defines the class but chooses no options, the document uses the default options for that class.
			\begin{itemize}
				\item The slide after next lists common options
			\end{itemize}
	\end{itemize}
\end{slide}
%================================Slide 8
\begin{slide}{Class (con't)}
	\begin{itemize}
		\item There are a handful of common classes:
			\begin{center}
				\begin{tabular}{l l}
					\texttt{article} & for short/medium documents with title and section headers. This class should be used for most papers. \\
					\texttt{book} & for books with chapters and title page. \\
					\texttt{report} & for shorter books (e.g. PhD. Thesis) \\
					\texttt{slides} & for presentations (e.g. this tutorial) \\
					\texttt{letter} & for professional/personal letters \\
				\end{tabular}
			\end{center}
		\item Only one class can be used for each document.
		\item There are a lot of other classes, Google for ``classes, LaTeX''
	\end{itemize}
\end{slide}
%================================Slide 8
\begin{slide}{Class Options}
	\begin{itemize}
		\item Each class can be slightly modified by using options. Below are options for the \texttt{article} class (defaults are in bold).
			\begin{center}
				\begin{tabular}{l l}
					\texttt{\textbf{10pt},11pt,12pt} & Font for the entire document. \\
					\texttt{\textbf{letterpaper},legalpaper,a4paper,a5paper} & The dimensions of the document. \\
					\texttt{\textbf{onecolumn},twocolumn} & The number of columns on the page. \\
					\texttt{\textbf{oneside},twoside} & Whether the layout of the pages are for single--sided or two--sided printing. \\
					\multicolumn{2}{c}{NOTE: This only adjusts the layout, it does not tell the printer to print two--sided.} \\
					\texttt{\textbf{portrait},landscape} & Either a vertical or horizontal orientation of the page. \\
				\end{tabular}
			\end{center}
	\end{itemize}
\end{slide}
%================================Slide 9
\begin{slide}{Class Options (con't)}
	\begin{itemize}
		\item Unlike classes themselves, an author may use many classes, for example, she may want to use A4 paper and make it two--sided.
		\item The author may separate several packages with a comma (,):
			\begin{center}
			\texttt{$\backslash$documentclass[a4,twoside]\{article\}}
			\end{center}
		\item Remember that only one class may be used for a document.
		\item However, the author may use as many options as she wishes, as long as they are properly separated in the command.
	\end{itemize}
\end{slide}
%================================Slide 10
\begin{slide}{Review of Classes}
	\begin{itemize}
		\item Remember the first line of every document should be \texttt{$\backslash$documentclass[\textit{options}]\{article\}}
		\item The author does not have to define the option, but does have to define the class.
		\item Many people have created their own classes, Google for ``class, LaTeX'' for some examples. Undergraduate and Graduate Deparments often create their own class to meet certain spacing/margin requirements.
	\end{itemize}
\end{slide}
%================================Slide 11
\begin{slide}{Packages}
	\begin{itemize}
		\item Packages load features for a document you are typing.
		\item For instance, you can use the \texttt{multicol} package so a document layout has multiple columns.
		\item We can also use packa
		\item Packages are not as important as classes, since they are \emph{features}.
		\item The \textit{packages} command should follow the \textit{document class} command:
			\begin{center}
			$\backslash$documentclass[\textit{options}]\{\textit{package}\}
			\end{center}
		\item Packages load features into our document we may want to use.
			\begin{itemize}
				\item Ex: Fancy headers, slide backgrounds, bibliography styles (MLA, APA)
			\end{itemize}
		\item Unlike the \texttt{$\backslash$documentclass}, packages are not required.
			\begin{itemize}
				\item Depends on what we want to include in our document.
			\end{itemize}
	\end{itemize}
\end{slide}
%================================Slide 12
\begin{slide}{Packages (con't)}
	\begin{itemize}
		\item However, similar to $\backslash$documentclass, we do not need to always use options for a package.
		\item If we wanted to use the \texttt{graphicx} package (don't worry what it does yet):
			\begin{center}
			\texttt{$\backslash$usepackage\{graphicx\}}
			\end{center}
		\item We may want to use the dvips option too:
			\begin{center}
			\texttt{$\backslash$usepackage[dvips]\{graphicx\}}
			\end{center}
	\end{itemize}
\end{slide}
%================================Slide 13
\begin{slide}{Package Options}
	\begin{itemize}
		\item There are a lot of packages and options, so we really can't list many of the useful ones.
		\item We will introduce applicable packages related to topics we encounter during this tutorial.
		\item It is likely a document will use multiple packages (this one does), there are two ways to list them:
			\begin{center}
				\begin{tabular}{l l}
				Acceptable: & \texttt{$\backslash$usepackage\{graphicx,fancybox\}} \\
				Acceptable: & \texttt{$\backslash$usepackage\{graphicx\}} \\
										& \texttt{$\backslash$usepackage\{fancybox\}} \\
				Not Acceptable:	&	\texttt{$\backslash$usepackage[dvips]\{graphicx,fancybox\}} \\
				Acceptable: & \texttt{$\backslash$usepackage[dvips]\{graphicx\}} \\
										& \texttt{$\backslash$usepackage\{fancybox\}}
				\end{tabular}
			\end{center}
		\item You may not invoke packages when listing several packages in one line.
	\end{itemize}
\end{slide}
%================================Slide 14
\begin{slide}{Packages Review}
	\begin{itemize}
		\item Packages enable features for our documents we may want to use.
		\item The package command(s) come immediately after the \texttt{$\backslash$documentclass} command.
		\item Multiple packages for a single document can be loaded in a single line or multiple lines.
		\item We will introduce useful packages as we explore several features in \LaTeX.
	\end{itemize}
\end{slide}
%================================Slide 15
\begin{slide}{Preamble Review}
	\begin{itemize}
 		\item The preamble of the document is from \texttt{$\backslash$documentclass} to \texttt{$\backslash$begin\{document\}}.
	 	\item \texttt{$\backslash$documentclass} tells \LaTeX\mbox{} the type of document it wants to be.
 		\item \texttt{$\backslash$usepackage} loads features we want to use for our document.
 		\item \texttt{$\backslash$documentclass} \emph{must} be use for each document, \texttt{$\backslash$usepackage} is optional.
 		\item The preamble will look similar to the one shown below:
			\begin{center}
		 		\texttt{$\backslash$documentclass[pdf]\{article\}} \\
			 	\texttt{$\backslash$usepackage\{graphicx,fancybox\}} \\
				\texttt{$\backslash$usepackage\{chem\}} \\
				\texttt{$\backslash$begin\{document\}} \\
			\end{center}
		\item NEXT: The body of the document.
	\end{itemize}
\end{slide}
\end{document}
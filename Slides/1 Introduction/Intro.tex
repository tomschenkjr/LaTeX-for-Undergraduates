%PLEASE READ=============================
%This document is copyrighted by Tom Schenk Jr and Drake University (2005). It may be modified and re-compiled for personal use only. A re-compiled version may not be distributed without permission of the copyrighters.
%Authors modifying this document for LaTeX for Undergraduates may append their name(s) in the ``\author'' command. They may not delete the names of previous authors.
\documentclass[pdf]{prosper}
%================================Packages
\usepackage{hyperref}
\usepackage{mflogo}
%================================Document Info./Slide 1
\title{Introduction}
\subtitle{Introduction to \LaTeX}
\author{Tom Schenk Jr.}
\institution{Drake University}
\email{tls007@drake.edu}
%================================Document Body
\begin{document}
\maketitle
%================================Slide 2
\begin{slide}{This Presentation}
	\begin{itemize}
		\item Look at the conceptual purpose of \LaTeX.
		\item Etymology of \LaTeX.
		\item Formatting versus Logical Design.
		\item History of \LaTeX.
	\end{itemize}
\end{slide}
%================================Slide 3
\begin{slide}{What is \LaTeX\?}
	\begin{itemize}
		\item An open-source typesetting program designed for preparing documents on PC or Mac.
		\item Free to use and easy to modify the source code for desired output.
		\item We will just focus on preparing documents for research papers and short assignments.
		\item Can also prepare books, slides, letters, and resume's.
		\item A sample document is on the webpage, snippets of code are later in the slide.
	\end{itemize}
\end{slide}
%================================Slide 4
\begin{slide}{What's in the Name?}
	\begin{itemize}
		\item \LaTeX\mbox{} is pronounced ``Lay-tech.''
		\item The name is actually a combination of two names:
			\begin{itemize}
				\item \TeX\mbox{} is the legacy typesetting program developed by Donald Knuth.
				\item ``La'' is derived from Leslie Lamport, the creator of \LaTeX.
			\end{itemize}
		\item \TeX\ stands for Tau-Epsilon-Chi.
		\item More on the etymology and history of \LaTeX\ and their creators later in this presentation.
	\end{itemize}
\end{slide}
%================================Slide 5
\begin{slide}{Why \LaTeX?}
	\begin{itemize}
		\item Microsoft Word, Corel Wordperfect, and OpenOffice are already word processing programs, why use \LaTeX?
		\item \LaTeX\ is a typesetting program.
			\begin{itemize}
				\item Logical Design versus Word Processing
			\end{itemize}
		\item Much better at quickly typing mathematical equations.			
			\begin{itemize}
				\item Math environment versus Equation Editor
			\end{itemize}
		\item Powerful enumerating, bibliography, indexing, table of contents tool
			\begin{itemize}
				\item Automatically generates and lists a bibliography at the end of the document.
			\end{itemize}
		\item Easily outputs to PostScript and PDF, which are both standards.
	\end {itemize}
\end{slide}
%================================Slide 6
\begin{slide}{The Downside}
	\begin{itemize}
		\item Word, Wordperfect, OpenOffice are WYSIWYG (What you see is what you get) programs.
			\begin{itemize}
				\item The layout of the screen roughly matches what is printed out.
			\end{itemize}
		\item \LaTeX\ resembles HTML or linear programming---which means there is a learning curve to use it.
		\item \LaTeX\ must be installed on the computer to \textit{prepare} documents.
		\item We will begin to focus on the design of documents.
	\end{itemize}
\end{slide}
%================================Slide 7
\begin{slide}{Logical Design vs Formatting}
	\begin{itemize}
		\item Word, Wordperfect, OpenOffice are WYSIWYG (What you see is what you get) programs.
			\begin{itemize}
				\item What appears on the screen is what appears on the printout.
			\end{itemize}
		\item To create titles, sections, or table of contents we must bold, italicize, or underline text.
			\begin{itemize}
				\item This process is known as \textit{formatting}.
			\end{itemize}
		\item Traditionally, we are use to these programs; however, formatting can be lengthy individual steps.
	\end{itemize}
\end{slide}
%================================Slide 8
\begin{slide}{Formatting}
	\begin{itemize}
		\item	Let's look at the process of formatting:
	\begin{center}
		\textbf{Walraisian Auctioneer in Asymmetric Markets}\\
		\textit{David E. Smith}\\
		February 19, 2004\\
	\end{center}
		\item In the prior paragraph those lines would require the author to \textbf{bold} the title, \textit{italicize} the name, and change it back to normal for the date.
		\item This may seem trite for a few lines, but can be time consuming for larger documents.
	\end{itemize}
\end{slide}
%================================Slide 9
\begin{slide}{Logical Design}
	\begin{itemize}
		\item The following is the header written in \LaTeX\ \\
		\texttt{$\backslash$documentclass\{article\}}\\
		\texttt{$\backslash$begin\{document\}}\\
		\texttt{$\backslash$textbf\{Walraisian Auctioneer in Asymmetric Markets\}}\\
		\texttt{$\backslash$textit\{David E. Smith\}}\\
		\texttt{February 19, 2004}
	\end{itemize}
\end{slide}
%================================Slide 10
\begin{slide}{Math Equations}
	\begin{itemize}
		\item Using mathematical notation in a WYSIWYG system is very time-consuming.
		\item Typesetting math equations are easy since \LaTeX\ uses commands.
		\item For instance, below is an example of the proper notation for the derivative of a logarithm.\\
		\texttt{$\backslash$frac\{dy\}\{dx\}=$\backslash$frac\{1\}\{x\}}\\
		\item The output is shown on the next slide.		
	\end{itemize}
\end{slide}
%================================Slide 11
\begin{slide}{Math Equations and Enumerating}
	\begin{itemize}
		\item Many papers also require that math equations are numbered.
\begin{center}
ex. The derivative of a log is\\
\begin{equation}
	\frac{dy}{dx}=\frac{1}{x}
\end{equation}
\end{center}
		\item It's easy to imagine where we need to insert another equation before (1).
		\item \LaTeX automatically enumerates equations so we don't have to manually change the sequence.
		\item If we have referenced equation (1) in the text, it will also renumber the references too.
	\end{itemize}
\end{slide}
%================================Slide 12
\begin{slide}{History of \LaTeX}
	\begin{itemize}
		\item Donald Knuth was researching for the fourth volume of the \textit{The Art of Computer Programming}.
		\item Upon reviewing the pre-prints for his manuscripts, he was disappointed in the typesetting.
		\item His initial reaction was ``bleech!'', which is phonically similar to \TeX.
		\item He stopped submitting papers to the American Mathematical Society (AMS).
	\end{itemize}
\end{slide}
%================================Slide 13
\begin{slide}{History of \LaTeX(con't)}
	\begin{itemize}
		\item In 1977 instead of researching in South America, he spend his time creating a new typesetting program at Stanford.
		\item The result was two programs: \TeX\ and \MF.
		\item The current version of \TeX\ is commonly known as \TeX 82.
		\item Official development of \TeX\ was frozen by Mr. Knuth (Addison-Wesley also owns \TeX).
		\item The version number is 3.14159 and converges to $\pi$.
		\item \MF\ is at 2.718 and converges to \textit{e}.
	\end{itemize}
\end{slide}
%================================Slide 14
\begin{slide}{History of \LaTeX\ (con't)}
	\begin{itemize}
		\item Donald Knuth encourages development of extensions to the \TeX\mbox{} program.
		\item Although he is widely known for \TeX\mbox{} and his book, \textit{The Art of Computer Programming}, he is also widely known for giving money who find errors in the \TeX\mbox{} source code or his books.
		\item \$2.56 per previously unknown mistake, although many people just save the checks.
			\begin{itemize}
				\item His website \href{http://www-cs-faculty.stanford.edu/\~knuth} lists the current errata list.
			\end{itemize}
	\end{itemize}
\end{slide}
%================================Slide 15
\begin{slide}{Creation of \LaTeX}
	\begin{itemize}
		\item Leslie Lamport developed a \TeX\mbox{} macro designed for easier document creation---she named it \LaTeX.
		\item During the 80's several version of \LaTeX\mbox{} were released. Many were incompatible with each other.
			\begin{itemize}
				\item Each version had its own specialty (e.g. math, slides, PDF generation)
			\end{itemize}
		\item In June 1994 \LaTeXe\mbox{} was released with the purpose of unifying the multitude of \LaTeX\mbox{} programs.
		\item Next version of \LaTeX\mbox{} is \LaTeX3.
	\end{itemize}
\end{slide}
%================================Slide 16
\begin{slide}{Conclusions}
	\begin{itemize}
		\item \LaTeX\mbox{} is a typesetting program designed to create documents using logical design.
		\item It can output to PostScript and PDF, even on Windows.
		\item It can automatically enumerate equations and section headers.
		\item NEXT: We will look at the broad process of how \LaTeX\mbox{} documents are created.
		\item LATER: We will show what is needed to install \LaTeX\mbox{} and create a simple document.
	\end{itemize}
\end{slide}

\end{document}


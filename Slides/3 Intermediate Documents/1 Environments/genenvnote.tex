%PLEASE READ=============================
%This document is copyrighted by Tom Schenk Jr and Drake University (2005). It may be modified and re-compiled for personal use only. A re-compiled version may not be distributed without permission of the copyrighters.
%Authors modifying this document for LaTeX for Undergraduates may append their name(s) in the ``\author'' command. They may not delete the names of previous authors.
\documentclass{article}
%================================Packages
\usepackage{graphicx}
\usepackage{verbatim}   % useful for program listings
\usepackage{color}      % use if color is used in text
\usepackage{subfigure}  % use for side-by-side figures

\usepackage{amsmath}    % need for subequations
%================================Document Info.
\title{\textsc{\LaTeX\ for Undergraduates\\
			General Environments} \\
			Lecture Notes}
\author{Tom Schenk Jr.}
\date{\textit{Version of \today}}
%================================Body
\begin{document}

\maketitle

\section{Motivation}

Some of the more advanced features in \LaTeX\ are enabled by environments---such as math equations. Environments are not terribly difficult, you have already seen a couple: bulleted and enumerated lists. In this lecture, a few basic environments will be reviewed before advancing to more interesting possibilities.

\section{Environments}

An environment is text and/or commands between \texttt{$\backslash$begin\{\textit{environment name}\}} and \texttt{$\backslash$end\{\textit{environment name}\}}. If you recall from Lecture 2.2.2, a bulleted list is an example of an environment.
\begin{verbatim}
\begin{itemize}
\item First item.
\item Second item.
\item Then some more\ldots
\end{itemize}
\end{verbatim}
Its output:
\begin{itemize}
\item First item.
\item Second item.
\item Then some more\ldots
\end{itemize}

Everything between begin and end will be formatted depending on the environment name. A few environments are discussed below.

\subsection{Quotation and Verse}

The \texttt{quotation} environment is the same as the \texttt{quote} environment from Lecture 2.2.2's lecture notes; however, it will be reiterated here.

Again, stressing the important of logical design, creating block quotation is left up to \LaTeX\ instead of the user. The \texttt{quotation} environment will indent the text from the left and right. For example,

\begin{verbatim}
\begin{quotation}
When once an efficient national government is established, the 
best mean in the country will not only consent to serve, but also
will enerally be appointed to manage for it\ldots.
\end{quotation}
\end{verbatim}
\begin{quotation}
When once an efficient national government is established, the 
best mean in the country will not only consent to serve, but also
will enerally be appointed to manage for it\ldots.
\end{quotation}

If you notice, the first paragraph is indented. You may offset that indentation by using the \texttt{$\backslash$noindent} command.
\begin{verbatim}
\begin{quotation}
\noindent
It adds no small weight to all these considerations to recollect
history informs us of no long--lived republic which had not a
senate. 
\end{quotation}
\end{verbatim}
\begin{quotation}
\noindent
It adds no small weight to all these considerations to recollect
history informs us of no long--lived republic which had not a
senate. 
\end{quotation}

One could also apply different justifications within an environment. Here the text is centered.
\begin{verbatim}
\begin{quotation}
\begin{center}
``Shall I compare thee to a summer's day?
Thou art more lovely and more temperate:
Rough winds do shake the darling buds of May,
And summer's lease hath all too short a date''.
\end{center}
\end{quotation}
\end{verbatim}
\begin{quotation}
\begin{center}
``Shall I compare thee to a summer's day?
Thou art more lovely and more temperate:
Rough winds do shake the darling buds of May,
And summer's lease hath all too short a date''.
\end{center}
\end{quotation}

Another pertinent environment is \texttt{verse}. Below is an example of its output.
\begin{verbatim}
\begin{verse}
``Shall I compare thee to a summer's day?
Thou art more lovely and more temperate:
Rough winds do shake the darling buds of May,
And summer's lease hath all too short a date''.
\end{verse}
\end{verbatim}
\begin{verse}
``Shall I compare thee to a summer's day?
Thou art more lovely and more temperate:
Rough winds do shake the darling buds of May,
And summer's lease hath all too short a date''.
\end{verse}

\subsection{Abstract}

Lastly, the \texttt{abstract} environment is one of the best. An abstract is a block of text at the beginning of a document that summarizes a research paper's topic and results. For published papers, the abstract may appear in the table of contents or as part of the citation in a bibliography.

\LaTeX\ will automatically format the text and identify the block of text as the abstract. Below is an example of an abstract:
\begin{verbatim}
\begin{abstract}
This study estimates the effect of unionization while 
controlling for other functions. Namely, this study 
attempts to answer two questions: (1) does unionization 
increase graduate assistant (GA) stipends and (2) does 
the extent of unionism contribute to wage increases? 
Using a data set of 2001--2002 stipends and an OLS 
regression, this study concludes that unionization does 
not impact stipend amounts. Similar to findings with 
faculty unions, wage differences are created by the 
student's discipline, region (i.e. cost--of--living), 
and institution type.
\end{abstract}
\end{verbatim}
\begin{abstract}
This study estimates the effect of unionization while 
controlling for other functions. Namely, this study 
attempts to answer two questions: (1) does unionization 
increase graduate assistant (GA) stipends and (2) does 
the extent of unionism contribute to wage increases? 
Using a data set of 2001--2002 stipends and an OLS 
regression, this study concludes that unionization does 
not impact stipend amounts. Similar to findings with 
faculty unions, wage differences are created by the 
student's discipline, region (i.e. cost--of--living), 
and institution type.
\end{abstract}

Normally, the abstract will be inserted right after \texttt{$\backslash$begin\{document\}}, \texttt{$\backslash$maketitle} or after the table of contents. If you place the abstract before \texttt{$\backslash$maketitle}, you will have an unexpected page break.

%================================Conclusion

\section{Conclusion}

This was a brief introduction to environments. The \texttt{quotation} environment will indent the text from the left and right. Including \texttt{$\backslash$noindent} will create a formal block quote. The \texttt{abstract} environment is very useful for research papers. However, I strongly recommend that one only uses this for an independent study paper or senior capstone.

\end{document}


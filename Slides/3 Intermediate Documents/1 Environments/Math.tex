\documentclass[pdf]{prosper}
%================================Packages
\usepackage{hyperref}
\usepackage{mflogo}
\usepackage{fancybox}
\usepackage{multicol}
%================================Document Info./Slide 1
\title{Environments}
\subtitle{Math Environment}
\author{Tom Schenk Jr.}
\institution{Drake University}
\email{tls007@drake.edu}
%================================Document Body
\begin{document}
\maketitle
%================================Slide 2
\begin{slide}{This Presentation}
	\begin{itemize}
		\item Basic math symbols
		\item \texttt{displaymath} and \texttt{equation} environment.
		\item Referencing equations.
		\item General math equations.
	\end{itemize}
\end{slide}
%================================Slide 2
\begin{slide}{Basic Math Symbols}
	\begin{itemize}
		\item In word processors, inserting math symbols/greek letters usually involves a few menus.
		\item \LaTeX\ uses command to make advanced symbols.
		\item We can use normal operators: $+$,$-$,$=$,$x$.
		\item For more complex operators, such as fractions, we use commands:
			\begin{center}
				$\backslash$frac\{\textit{numerator}\}\{\textit{denominator}\} \\
				$\frac{\textit{numerator}}{\textit{denominator}}$
			\end{center}
		\item The next slide contains a table of basic math operators.
	\end{itemize}
\end{slide}
%================================Slide 2
\begin{slide}{Basic Math Symbols Table}
	\begin{center}
		\begin{tabular}{l l l l}
			\hline \\
			Command & Output & Command & Output\\
			\hline \\
			\texttt{$\backslash$cdot} & $\cdot$ & \texttt{$\backslash$times} & $\times$ \\
			\texttt{$\backslash$div} & $\div$ & \texttt{$\backslash$equiv} & $\equiv$ \\ 
			\texttt{$\backslash$sqrt} & $\sqrt{}$ & \texttt{< >} & $<$ $>$ \\
			\texttt{$\backslash$leq} & $\leq$ & \texttt{$\backslash$geq} & $\geq$ \\
			\texttt{$\backslash$sum} & $\sum$ & \texttt{$\backslash$int} & $\int$ \\
		\end{tabular}
	\end{center}
\end{slide}
%================================Slide 2
\begin{slide}{Math Example}
	\texttt{$\backslash$frac\{$a$\}\{$b$\} $\backslash$cdot $\backslash$frac\{$c$\}\{$d$\}} \\
	$\frac{a}{b} \cdot \frac{c}{d}$
	\begin{itemize}
		\item Refer to the supplemental documentation for more lists of operators and symbols.
	\end{itemize}
\end{slide}
%================================Slide 2
\begin{slide}{Greek Characters}
	\begin{itemize}
		\item Greek letters can also be made by using simple commands.
		\item For instance, the letter Alpha ($\alpha$) can be made with \texttt{$\backslash$alpha}.
		\item \LaTeX\ recognizes the difference of upper case and lower case in the input code.
		\item For example, \texttt{$\backslash$xi} and \texttt{$\backslash$Xi} produces $\xi$ and $\Xi$, respectively.
		\item The supplemental documents contain the list of Greek characters.
	\end{itemize}
\end{slide}
%================================Slide 2
\begin{slide}{Super and Subscripts}
	\begin{itemize}
		\item To make a superscript we use the following command: \\
		\texttt{$\backslash$$a$\^{}\{$x$\}} will produce $a^{x}$.
		\item To make a subscript we use the following command: \\
		\texttt{$\backslash$$\beta$\_\{$1$\}} will produce $\beta_{1}$ \\
	\end{itemize}
\end{slide}
%================================Slide 2
\begin{slide}{Displaymath Environment}
	\begin{itemize}
		\item One of the many math environments is the \texttt{displaymath} environment.
		\item This environment will give an equation some vertical spacing between lines.
		\item This environment is especially helpful if we're trying to point out an equation in a document.
	\end{itemize}
\end{slide}
%================================Slide 2
\begin{slide}{Displaymath Environment Example}
	\texttt{Here we have some very basic text that would appear in a document. Then we want to introduce an equation, like the one below:} \\
	\texttt{$\backslash$begin\{displaymath\}} \\
	\texttt{E=MC\^{}\{2\}} \\
	\texttt{$\backslash$end\{displaymath\}} \\
	\texttt{Now we return to normal text. We can see how the equation is easy to read.}
\end{slide}
%================================Slide 2
\begin{slide}{Displaymath Environment Example (con't)}
	Here we have some very basic text that would appear in a document. Then we want to introduce an equation, like the one below:
	\begin{displaymath}
	E=MC^{2}
	\end{displaymath}
	Now we return to normal text. We can see how the equation is easy to read.
\end{slide}
%================================Slide 2
\begin{slide}{Displaymath Alternative}
	\begin{itemize}
		\item We can use an alternative to the \texttt{displaymath} environment that has the same output, but without having to type the cumbersome environment commands.
		\item Placing \texttt{$\backslash$[} and \texttt{$\backslash$]} around an equation will make it display the exact same way as using the \texttt{displaymath} environment.
		\item This takes a lot less time with no difference to the reader.
		\item If you prefer to keep a very organized input source file, then using environments will be a much cleaner look.
	\end{itemize}
\end{slide}
%================================Slide 2
\begin{slide}{Displaymath Alternative Example}
	\texttt{Let's return to a similar example to the one above. Here we will use the shorter command. Remember, this command is meant to make an equation appear with a lot of space around it, such as the one below:} \\
	\texttt{$\backslash$[E=MC\^{}\{2\}$\backslash$]} \\
	\texttt{Now we return to normal text. We can see how the equation is easy to read. Also, we dropped two lines from the code.}
\end{slide}
%================================Slide 2
\begin{slide}{Displaymath Alternative Example (con't)}
	Let's return to a similar example to the one above. Here we will use the shorter command. Remember, this command is meant to make an equation appear with a lot of space around it, such as the one below:
	\[E=MC^{2}\]
	Now we return to normal text. We can see how the equation is easy to read. Also, we dropped two lines from the code.
\end{slide}
%================================Slide 2
\begin{slide}{Equation Environment}
	\begin{itemize}
		\item In academic journals mathematical equations are often numberd either to the left or right of the equation.
		\item This allows authors to reference by number.
		\item This involves a little bit of formatting in word processors, and even when it's done, it does not automatically renumber the equations.
		\item To create a numbered equation, we can use the \texttt{equation} environment.
		\item The equation will appear the same way they do in \texttt{displaymath}, except it is numbered.
	\end{itemize}
\end{slide}
%================================Slide 2
\begin{slide}{Equation Environment Example}
	\texttt{In an academic journal, we may want to introduce an important equation and have it numbered for easy reference. So we will induce the equation using the equation environment:} \\
	\texttt{$\backslash$begin\{equation\}} \\
	\texttt{y=$\backslash$alpha+$\backslash$beta\_\{n\}x} \\
	\texttt{$\backslash$end\{equation\}} \\
	\texttt{Now we return to normal text and continue the article. We can also reference the above equation by number.}
\end{slide}
%================================Slide 2
\begin{slide}{Equation Environment Example (con't)}
	In an academic journal, we may want to introduce an important equation and have it numbered for easy reference. So we will induce the equation using the equation environment:
	\begin{equation}
	y=\alpha+\beta_{n}x
	\end{equation}
	Now we return to normal text and continue the article. We can also reference the above equation by number.
\end{slide}
%================================Slide 2
\begin{slide}{More on Equation Environment}
	\begin{itemize}
		\item If we insert an equation prior to the first equation, then \LaTeX\ will automatically renumber all of the equations.
		\item Also, \LaTeX\ has an easy way to reference equations.
		\item Putting the actual equation number in the text isn't efficient, if the equation number changed, then we would manually need to change it.
		\item So, we can label equations and then refer to that equation by name. When the equation numbers change, our corresponding references would also change.
		\item Next to \texttt{$\backslash$begin\{environment\}}, we place \texttt{$\backslash$label\{eq:\textit{name}\}}
		\item We then refer to that in our text with either with \texttt{($\backslash$ref\{eq:\textit{name}\})} or \texttt{$\backslash$eqref\{eq:\textit{name}\}}.
	\end{itemize}
\end{slide}
%================================Slide 2
\begin{slide}{More on Equation Environment Example}
	\texttt{Let's return to the equation example. Now we want to have our equation numbered and labeled so we can refer to it.}
	\texttt{$\backslash$begin\{equation\}} \texttt{$\backslash$label\{eq:\textit{line}\}} \\
	\texttt{y=$\backslash$alpha+$\backslash$beta\_\{n\}x} \\
	\texttt{$\backslash$end\{equation\}} \\
	\texttt{Now we return to normal text and continue the article. As you can see, the above equation is labeled \texttt{($\backslash$ref\{eq:\textit{line}\})}}.
\end{slide}
%================================Slide 2
\begin{slide}{More on Equation Environment Example (con't)}
	Let's return to the equation example. Now we want to have our equation numbered and labeled so we can refer to it.
	\begin{equation} \label{eq:line}
	y=\alpha+\beta_{n}x
	\end{equation}
	Now we return to normal text and continue the article. As you can see, the above equation is labeled (\ref{eq:line}).
\end{slide}
%================================Slide 2
\begin{slide}{General Equations}
	\begin{itemize}
		\item \texttt{displaymath} and \texttt{equation} environments inserts space between the equation and the body text.
		\item Sometimes it's appropriate to include the equations within a sentence without extra spacing.
		\item Within the body of the text, we can use \texttt{\$} to start and end an equation.
		\item This is especially helpful when we're just showing an equation after some algebraic manipulation.
	\end{itemize}
\end{slide}
%================================Slide 2
\begin{slide}{General Equation Example}
	\texttt{Below we present the Pythagorean theorem in it's common form:}
	\texttt{$\backslash$begin\{displaymath\}} \\
	\texttt{a\^{}\{2\}+b\^{}\{2\}=c\^{}\{2\}} \\
	\texttt{$\backslash$end\{displaymath\}} \\
	\texttt{But there are several forms, we can also look at it as \$$\backslash$sqrt\{a\^{}\{2\}+b\^{}\{2\}\}=c\$. We can also reference the above equation by number.}
\end{slide}
%================================Slide 2
\begin{slide}{General Equation Example (con't)}
	Below we present the Pythagorean theorem in it's common form: \\
	\begin{displaymath}
	a^{2}+b^{2}=c^{2}
	\end{displaymath}
	But there are several forms, we can also look at it as $\sqrt{a^{2}+b^{2}}=c$.
\end{slide}
%================================Slide 2
\begin{slide}{Matrices}
	\begin{itemize}
		\item When we're using the \texttt{displaymath} and \texttt{equation} environment we can make nice looking matrices.
		\item Within our environment we need to include another command called \texttt{$\backslash$begin\{array\}} to begin the actual matrix
		\item Next to the beginning of the array, we include the justification of our matrix, for matrices we always use \texttt{\{ccc\}} to center the numbers in the columns.
		\item Look closely at the following example to see how matrices work.
	\end{itemize}
\end{slide}
%================================Slide 2
\begin{slide}{Matrix Example}
	\texttt{$\backslash$begin\{displaymath\}} \\
	\texttt{$\backslash$left( $\backslash$begin\{array\}\{ccc\} $\backslash$$\backslash$} \\
	\texttt{x\_\{11\} \& x\_\{12\} \& x\_\{1i\} $\backslash$$\backslash$} \\
	\texttt{x\_\{21\} \& x\_\{22\} \& x\_\{2i\} $\backslash$$\backslash$} \\
	\texttt{x\_\{j1\} \& x\_\{j2\} \& x\_\{ji\}} \\
	\texttt{$\backslash$end\{array\} $\backslash$right)} \\
	\texttt{$\backslash$end\{displaymath\}}
\end{slide}
%================================Slide 2
\begin{slide}{Matrix Example (con't)}
	\begin{displaymath}
	\left( \begin{array}{ccc}
	x_{11} & x_{12} & x_{1i} \\
	x_{21} & x_{22} & x_{2i} \\
	x_{j1} & x_{j2} & x_{ji}
	\end{array} \right)
	\end{displaymath}
\end{slide}
%================================Slide 2
\begin{slide}{Math Environment Review}
	\begin{itemize}
		\item The \texttt{displaymath} environment inserts an equation with lots of space around it.
		\item The \texttt{equation} environment is the same as the \texttt{displaymath}, but numbers each equation.
		\item To make equations within the text, begin and end the formula with \$.
		\item Within a math environment, we can make a matrix using \texttt{$\backslash$begin\{array\}}.
		\item Take a look at the supplemental documents for lists of commands and extensive math formatting.
		\item NEXT: Tables
	\end{itemize}
\end{slide}

\end{document}
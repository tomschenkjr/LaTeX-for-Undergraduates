\documentclass[pdf]{prosper}
%================================Packages
\usepackage{hyperref}
\usepackage{mflogo}
\usepackage{fancybox}
\usepackage{multicol}
%================================Document Info./Slide 1
\title{Indexing and Bibliographies}
\subtitle{Indexing}
\author{Tom Schenk Jr.}
\institution{Drake University}
\email{tls007@drake.edu}
%================================Document Body
\begin{document}
\maketitle
%================================Slide 2
\begin{slide}{This Presentation}
	\begin{itemize}
		\item Introducing the \texttt{makeidx} package.
		\item The \texttt{$\backslash$makeindex} command.
		\item Creating index points.
		\item Creating the index.
	\end{itemize}
\end{slide}
%================================Slide 2
\begin{slide}{Overview of Indexing}
	\begin{itemize}
		\item Often at the end of a book, an index is included with references to key point.
		\item Usually this would include a separate piece of paper keeping track of words and their page of occurence.
		\item If pages were included or excluded, then it would be necessary to adjust all of the enteries.
		\item We can automate this process using the \texttt{makeidx} package.
	\end{itemize}
\end{slide}
%================================Slide 2
\begin{slide}{makeidx Package}
	\begin{itemize}
		\item To include the \texttt{makeidx} package, insert \texttt{$\backslash$usepackage\{makeidx\}} in the preamble.
		\item We also need to enable the package by including \texttt{$\backslash$makeindex} in the preamble. \\
\texttt{$\backslash$documentclass\{book\}} \\
\texttt{$\backslash$usepackage\{makeidx\}} \\
\texttt{$\backslash$title\{Senior Capstone Project\}} \\
\texttt{$\backslash$author\{Carl Limyao\}} \\
\texttt{$\backslash$makeindex} \\
\texttt{$\backslash$begin\{document\}} \\
	\end{itemize}
\end{slide}
%================================Slide 2
\begin{slide}{Making Index Points}
	\begin{itemize}
		\item To create an index point---a point where a word is noted and placed in the index---we use \texttt{$\backslash$index\{\textit{key}\}}.
		\item We need to include the index command after the word itself.
			\begin{itemize}
				\item e.g. \texttt{Maxwell's equations$\backslash$\{Maxwell\} are four equations\ldots}
			\end{itemize}
		\item Some entries are sub--entries of other index words. To do this we use \texttt{$\backslash$index\{\textit{key!sub--key}\}}
	\end{itemize}
\end{slide}
%================================Slide 2
\begin{slide}{Creating the Index}
	\begin{itemize}
		\item To generate the index, use \texttt{$\backslash$printindex} toward the end of the document.
		\item The index will be generated in the place where the command is placed.
		\item For general practice, it is best to place the index after the bibliography.
		\item When generating the document, the index will be put in the output file (DVI, PS, PDF) and a .idx file with the same name as our .tex document.
		\item The .idx file has all of our index listings.
	\end{itemize}
\end{slide}
%================================Slide 2
\begin{slide}{Index Review}
	\begin{itemize}
		\item We use the \texttt{makeidx} package to make indexes.
		\item It's necessary to include \texttt{$\backslash$makeindex} in the preamble.
		\item Index entires are categorized using \texttt{$\backslash$index\{\textit{key}\}}.
		\item Toward the end of the document, use \texttt{$\backslash$printindex} to make the index.
	\end{itemize}
\end{slide}
\end{document}
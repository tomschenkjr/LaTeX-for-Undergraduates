%READ BEFORE EDITING=======================DO NOT DELETE==========================
%Author(s): Tom Schenk Jr.
%Last Revision: 10/25/2005
%Copyright: Licensed under Creative Commons Attribution-NonCommercial-ShareAlike 2.5 License
%Copyright URI: http://creativecommons.org/licenses/by-nc-sa/2.5/
%You may edit and distribute this document for non-commercial services with attributions to previous authors. You may add your name to the \author{} command if you contributed to this document.

%================================Slides
\documentclass[pdf]{prosper}
%================================Packages
\usepackage{hyperref}
\usepackage{mflogo}
\usepackage{fancybox}
%================================Document Info./Slide 1
\title{Preamble}
\subtitle{Classes and Packages}
\author{Tom Schenk Jr.}		%Add your name here if you modified this document. (e.g. \author{name1, name2)
\institution{Drake University}
%================================Document Body
\begin{document}
\maketitle
%================================Slide 2
\begin{slide}{This Presentation}
	\begin{itemize} 
		\item Structure of \LaTeX\mbox{} Documents
		\item Preamble
		\item Classes
		\item Class Options
		\item Packages
		\item Package Options
	\end{itemize}
\end{slide}
%================================Slide 3
\begin{slide}{Structure of \LaTeX\mbox{} document}
	\begin{itemize}
		\item Let's focus on the source file for documents (.tex).
		\item Every file has the same structure.
		\item A document is composed of a preamble and a body.
		\item The preamble tells \LaTeX\mbox{} how the document is going to look (e.g. font size, margins) and the body is the actual text.
		\item We will concentrate on the preamble in these next two presentations.
		\item This presentation will look at Classes and Packages.
	\end{itemize}
\end{slide}
%================================Slide 4
\begin{slide}{Structure of \LaTeX\mbox{} document}
	\begin{quote}
		\texttt{$\backslash$documentclass[pdf]\{article\}} \\
		\texttt{$\backslash$usepackage\{hhref\}} \\
		\texttt{$\backslash$title\{Exploitation and Unfreedom\}}
		\texttt{$\backslash$author\{Trin Turner\}} \\
		\texttt{$\backslash$institution\{Drake University\}} \\
		\texttt{$\backslash$begin\{document\}} \\
		\texttt{$\backslash$maketitle} \\
		\texttt{Some part of the body...} \\
		\texttt{$\backslash$end{document}} \\
	\end{quote}
	\begin{itemize}
		\item Above is a generic layout of \LaTeX\mbox{} commands for a very basic document.
	\end{itemize}
\end{slide}
%================================Slide 5
\begin{slide}{Preamble}
	\begin{itemize}
		\item The preamble is the lines of text before \texttt{$\backslash$begin\{document\}}.
		\item Preamble contains information that will be used by \LaTeX\mbox{} when it generates a document.
		\item The first two lines of a document are \texttt{$\backslash$documentclass} and \texttt{$\backslash$usepackage} commands.
		\item \texttt{$\backslash$documentclass} tells \LaTeX\mbox{} what kind of document you are creating.
		\item \texttt{$\backslash$usepackage} will load features you want to enable for the document.
		\item The preamble can also contains other information, such as the title and author, along with the pages margins (covered in the next presentation).
	\end{itemize}
\end{slide}
%================================Slide 6
\begin{slide}{Classes}
	\begin{itemize}
		\item Classes inform \LaTeX\mbox{} what kind of document you are creating.
		\item This is important because the layout of one type of document (e.g. article) is different from another (e.g. book)
		\item The Class command also lets you change the default options for a document (e.g. font size, paper format)
		\item The first line of every document should be the Class command\footnote{Actually, the author may insert comments that will not be read by \LaTeX\mbox{} before the Class command. Beginning a line with \% will tell \LaTeX\mbox{} to ignore everything on that line.}:
			\begin{center}			
				\texttt{$\backslash$documentclass[\textit{options}]\{\textit{class}\}}
			\end{center}
	\end{itemize}
\end{slide}
%================================Slide 7
\begin{slide}{Class}
	\begin{itemize}
		\item The Class command consists of two operators, \textit{class} and \textit{options}.
		\item \textit{Class} must be included because it specifically tells \LaTeX\mbox{} the document type.
			\begin{itemize}
				\item A list of common classes are listed on slide 9.
			\end{itemize}
		\item \textit{Options} lets the author choose her own default parameters for the document (e.g. font size).
			\begin{itemize}
				\item A list of common options are listed on slide 10.
			\end{itemize}
	\end{itemize}
\end{slide}
%================================Slide 8
\begin{slide}{Class (con't)}
	\begin{itemize}
		\item Using the \texttt{[\textit{options}]} command is optional, but the author \emph{must} define the class:
			\begin{center}
				\begin{tabular}{l l}
					Acceptable: & \texttt{$\backslash$documentclass\{article\}} \\
					Acceptable: & \texttt{$\backslash$documentclass[12pt]\{article\}} \\
					Not Acceptable: & \texttt{$\backslash$documentclass[12pt]}
				\end{tabular}
			\end{center}
		\item When the author defines the class but chooses no options, the document uses the default options for that class.
	\end{itemize}
\end{slide}
%================================Slide 9
\begin{slide}{List of Common Classes}
	\begin{itemize}
		\item There are a handful of common classes:
			\begin{center}
				\begin{tabular}{l l}
					\texttt{article} & Short/medium papers with sections. \\
					\texttt{book} & for books with chapters and title page. \\
					\texttt{report} & for shorter books (e.g. PhD. Thesis) \\
					\texttt{slides} & for presentations (e.g. this tutorial) \\
					\texttt{letter} & for professional/personal letters \\
				\end{tabular}
			\end{center}
		\item Only one class can be used for each document.
		\item Students should generally use the \texttt{article} class.
		\item There are a lot of other classes, Google for ``classes, LaTeX''
	\end{itemize}
\end{slide}
%================================Slide 10
\begin{slide}{List of Common Class Options}
	\begin{itemize}
		\item Each class can be slightly modified by using options. Below are options for the \texttt{article} class (defaults are in bold).
				\begin{tabular}{l p{2in} l}
					\texttt{\textbf{10pt},11pt,12pt} & \\ 
					\multicolumn{2}{l}{Font for the entire document.} \\
					\texttt{\textbf{letterpaper},legalpaper,a4paper,a5paper} & \\
					\multicolumn{2}{l}{The dimensions of the document.} \\
					\texttt{\textbf{onecolumn},twocolumn} & \\
					\multicolumn{2}{l}{The number of columns on the page.} \\
					\texttt{\textbf{oneside},twoside} & \\
					\multicolumn{2}{l}{Single--sided or two--sided print layout.} \\
					\texttt{\textbf{portrait},landscape} & \\
					\multicolumn{2}{l}{Either a vertical or horizontal orientation of the page.} \\
				\end{tabular}
	\end{itemize}
\end{slide}
%================================Slide 11
\begin{slide}{Class Options (con't)}
	\begin{itemize}
		\item Unlike classes themselves, an author may use many classes, for example, she may want to use A4 paper and make it two--sided.
		\item The author may separate several packages with a comma (,):
			\begin{center}
			\texttt{$\backslash$documentclass[a4,twoside]\{article\}}
			\end{center}
		\item Remember that only one class may be used for a document.
		\item However, the author may use as many options as she wishes, as long as they are properly separated in the command.
	\end{itemize}
\end{slide}
%================================Slide 12
\begin{slide}{Review of Classes}
	\begin{itemize}
		\item Remember the first line of every document should be \texttt{$\backslash$documentclass[\textit{options}]\{article\}}
		\item The author does not have to define the option, but does have to define the class.
		\item Many people have created their own classes, Google for ``class, LaTeX'' for some examples. Undergraduate and Graduate Deparments often create their own class to meet certain spacing/margin requirements.
	\end{itemize}
\end{slide}
%================================Slide 13
\begin{slide}{Packages}
	\begin{itemize}
		\item Packages load features for a document you are typing.
		\item For instance, you can use the \texttt{multicol} package so a document layout has multiple columns.
		\item We can also use packages to load fancy headers or bibliography styles.
		\item Packages are not as important as classes, since they are \emph{features}.
		\item The \textit{packages} command should follow the \textit{document class} command:
			\begin{center}
			\texttt{$\backslash$documentclass[\textit{options}]\{\textit{package}\}}
			\end{center}
		\item \textit{Packages} are not required for a document.
	\end{itemize}
\end{slide}
%================================Slide 14
\begin{slide}{Packages (con't)}
	\begin{itemize}
		\item Similar to $\backslash$documentclass, we do not always need to use options for a package.
		\item If we wanted to use the \texttt{graphicx} package (don't worry what it does yet):
			\begin{center}
			\texttt{$\backslash$usepackage\{graphicx\}}
			\end{center}
		\item We may want to use the dvips option too:
			\begin{center}
			\texttt{$\backslash$usepackage[dvips]\{graphicx\}}
			\end{center}
		\item Either command is acceptable, it depends on what you need for your document.
	\end{itemize}
\end{slide}
%================================Slide 15
\begin{slide}{Package Options}
	\begin{itemize}
		\item There are a lot of packages and options, so we really can't list many of the useful ones.
		\item We will introduce applicable packages related to topics we encounter during this tutorial.
		\item It is likely a document will use multiple packages (this one does), there are two ways to list them:
				\begin{tabular}{l l}
				Acceptable: & \texttt{$\backslash$usepackage\{babel,ams\}} \\
				Acceptable: & \texttt{$\backslash$usepackage\{babel\}} \\
										& \texttt{$\backslash$usepackage\{ams\}} \\
				Not Acceptable:	&	\texttt{$\backslash$usepackage[german]\{babel,ams\}} \\
				Acceptable: & \texttt{$\backslash$usepackage[german]\{babel\}} \\
										& \texttt{$\backslash$usepackage\{ams\}}
				\end{tabular}
	\end{itemize}
\end{slide}
%================================Slide 16
\begin{slide}{Packages Review}
	\begin{itemize}
		\item Packages enable features for our documents we may want to use.
		\item The package command(s) come immediately after the \texttt{$\backslash$documentclass} command.
		\item Multiple packages for a single document can be loaded in a single line or multiple lines.
		\item We will introduce useful packages as we explore several features in \LaTeX\mbox{}.
	\end{itemize}
\end{slide}
%================================Slide 17
\begin{slide}{Preamble Review}
	\begin{itemize}
 		\item The preamble of the document is from \texttt{$\backslash$documentclass} to \texttt{$\backslash$begin\{document\}}.
	 	\item \texttt{$\backslash$documentclass} tells \LaTeX\mbox{} the type of document it wants to be.
 		\item \texttt{$\backslash$usepackage} loads features we want to use for our document.
 		\item \texttt{$\backslash$documentclass} \emph{must} be use for each document, \texttt{$\backslash$usepackage} is optional.
 	\end{itemize}
\end{slide}
%================================Slide 18
\begin{slide}{Preamble Review (con't)}
 	\begin{itemize}
 		\item The preamble will look similar to the one shown below:
		 		\texttt{$\backslash$documentclass[pdf]\{article\}} \\
			 	\texttt{$\backslash$usepackage\{graphicx,fancybox\}} \\
				\texttt{$\backslash$usepackage\{chem\}} \\
				\texttt{$\backslash$begin\{document\}} \\
		\item NEXT: Title, Author, and Page Layout.
	\end{itemize}
\end{slide}
\end{document}
%READ BEFORE EDITING=======================DO NOT DELETE==========================
%Author(s): Tom Schenk Jr.
%Last Revision: 1/9/2006
%Copyright: Licensed under Creative Commons Attribution-NonCommercial-ShareAlike 2.5 License
%Copyright URI: http://creativecommons.org/licenses/by-nc-sa/2.5/
%You may edit and distribute this document for non-commercial services with attributions to previous authors. You may add your name to the \author{} command if you contributed to this document.

\documentclass{article}
%================================Packages
\usepackage{graphicx}
\usepackage{verbatim}   % useful for program listings
\usepackage{color}      % use if color is used in text
\usepackage{subfigure}  % use for side-by-side figures

\usepackage{amsmath}    % need for subequations
%================================Document Info.
\title{\textsc{\LaTeX\ for Undergraduates\\
			General Environments} \\
			Lecture Notes}
\author{Tom Schenk Jr.}		%Add your name here if you modified this document. (e.g. \author{name1, name2)
\date{\textit{Version of \today}}
%================================Body
\begin{document}

\maketitle

\section{Motivation}

Some of the more advanced features in \LaTeX\ are enabled by environments---such as math equations. 

You may have noticed a handful of operators are used to begin and end commands, for instance, \texttt{$\backslash$} and \{ \} are often used in the syntax. So what happens if we want to actually make those characters appear in the text and not function as operators (such as above)?

More importantly, how do we italicize, bold, or use different fonts in a document? As explained earlier, \LaTeX\ will take care of most of our formatting needs (e.g. section headers), nevertheless, it is still necessary to use traditional formatting throughout a document.

Lastly, this lecture will explore how to create bulleted or enumerate lists in our document. The material on lists is directly related to the next lecture on environments.

\section{Formatting}

\subsection{Special Characters}

A number of characters are ``special'' characters. That is, they are used in regular English syntax, but serve an important function in \LaTeX. I have already pointed out that $\backslash$ is one of these characters. Simply including it in our code will cause \LaTeX\ to misinterpret its meaning. Instead, we must use commands to make certain characters. Below is a list of characters and their corresponding codes:
\begin{tabular}{l l l l l l l l l}
				\texttt{$\backslash$\$} & \texttt{$\backslash$\%} & \texttt{$\backslash$\&} & \texttt{$\backslash$\#} & \texttt{$\backslash$\{} & \texttt{$\backslash$\}} & \texttt{$\backslash$\_} & \texttt{$\backslash$\^{}\{\}} & \texttt{$\backslash$\~{}\{\}} \\
				\$ & \% & \& & \# & \{ & \} & \_ & \^{} & \~{} \\
\end{tabular}
You may have noticed some of the special commands are followed with \{\}. Let us take a look at the carrot (\^{}) character as an example. The following is what ``apple'' should look like if properly proceeded by a carrot: \^{}apple (\texttt{$\backslash$\^\{\}apple}). Say, for instance, I omitted the \{\}, then it would appear as \^apple (\texttt{$\backslash$\^apple}). As you can see, the carrot then acts as an accent character. Implicitly, \{\} acts as an artificial space between \^{} and ``apple.'' The same is true for the tild\~e character.

Along with special formatting commands, there are special logo's that \LaTeX\ can recognize. Below is a list of commands and their respective outputs:
		\begin{table}\caption{Built--in Logo Commands}\label{tbl:logo}	
			\begin{center}
				\begin{tabular}{|l|l|}
					\hline
					Command & Logo \\
					\hline
					$\backslash$LaTeX$\backslash$ & \LaTeX\ \\
					\hline
					$\backslash$LaTeXe$\backslash$ & \LaTeXe\ \\
					\hline
					$\backslash$TeX$\backslash$ & \TeX\ \\
					\hline
				\end{tabular}
			\end{center}
		\end{table}
This is a nifty tool, but seldom used when not creating a \LaTeX\ tutorial. You may have noticed the use of an extra $\backslash$ at the end of the commands. Again, this is to solve a spacing problem. If you were not to include the extra backslash, then it would appear like this: \LaTeX Tutorial (\texttt{$\backslash$LaTeX Tutorial}).

\subsection{Formatting}

Formatting text is rather simple. \texttt{$\backslash$textbf\{\textit{argument}\}}, for instance, will produce \textbf{bold} text. \texttt{$\backslash$texttt\{\textit{argument}\}} will produce \texttt{typewriter} font; \texttt{$\backslash$textsc\{\textit{argument}\}} will use \textsc{small caps}; \texttt{$\backslash$textbf\{\textsl{argument}\}} will make the text appear \textsl{slanted}. \texttt{$\backslash$textit\{\textit{argument}\}} will produce \textit{italicized} font, but so will \texttt{$\backslash$emph\{\textit{argument}\}}---depending on the circumstances. Normally, italicizing a word is meant to emphasize its importance; meanwhile, italicizing is usually meant for design purposes.

Consider the following two paragraphs. Both use \texttt{$\backslash$emph} to emphasize a word. Although the second paragraph's default is italics, emphasized words are still contrasted.

\begin{quote}
A student may be absent up to \emph{six} times before being dismissed from class. Likewise, he or she may only be tardy \emph{twelve} times before also being dismissed. An absent or tardiness can be excused if he or she brings a valid note.
\end{quote}

\begin{quote}
\textit{A student may be absent up to \emph{six} times before being dismissed from class. Likewise, he or she may only be tardy \emph{twelve} times before also being dismissed. An absent or tardiness can be excused if he or she brings a valid note.}
\end{quote}

Figure \ref{tbl:format} outlines some basic formatting commands and their results.

\begin{table}[!h] \caption{Widely Used Formatting Commands} \label{tbl:format}
	\begin{center}
		\begin{tabular}{r l} 
		\hline
		Command & Output \\
		\hline
		$\backslash$textbf\{\textit{text}\} & \textbf{bold} \\
		$\backslash$textit\{\textit{text}\} & \textit{italic} \\
		$\backslash$texttt\{\textit{text}\} & \texttt{typewriter} \\
		$\backslash$textsl\{\textit{text}\} & \textsl{slanted} \\
		$\backslash$emph\{\textit{text}\} 	& \emph{emphasized} \\
		$\backslash$textsf\{\textit{text}\} & \textsf{sans serif} \\
		$\backslash$textsc\{\textit{text}\} & \textsc{small caps} \\ 
		\end{tabular}
	\end{center}		
\end{table}

\subsection{Font Sizes}

Recall from Lecture 2.1.1, an author may change the font size as an option in the \textit{document class} command (e.g. \texttt{$\backslash$documentclass[12pt]\{article\}}). Sometimes, one may want to temporarily change a font size. This may be done with a variety of commands. For instance, \texttt{$\backslash$tiny} will change all proceeding text to \tiny size 5 font \normalsize(note: there are no brackets with this command). On the other hand, \texttt{$\backslash$large} will create \large size 12 font. \normalsize To return to your default font size, you must turn off the command with \texttt{$\backslash$normalsize}. Consider the following example:

\begin{verbatim}
Here is an example of \footnotesize footnotesize font. 
\normalsize After returning to normal size, you may want to use a
\LARGE large font.
\end{verbatim}

And its output:

\begin{quote}
Here is an example of \footnotesize footnotesize font. \normalsize After returning to normal size, you may want to use a \LARGE large font.
\end{quote}

It should be noted that these commands are case sensitive. That is, \texttt{$\backslash$}LARGE is different from \texttt{$\backslash$Large}. Table \ref{tbl:font} shows font commands and their respective output. In general, each increase (decrease) in font size is +2 (-2) in point size.

\begin{table}[!h] \caption{Font Commands and Their Respective Output (default is 10pt)} \label{tbl:font}
	\begin{center}
		\begin{tabular}{r l} 
		\hline
		Command & Output \\
		\hline
					$\backslash$tiny & \tiny tiny font \\
					$\backslash$scriptsize & \scriptsize script size font \\
					$\backslash$footnotesize & \footnotesize footnote size font \\
					$\backslash$small & \small small font \\
					$\backslash$normalsize & \normalsize normal font \\
					$\backslash$large & \large large font \\
					$\backslash$Large & \large larger font \\
					$\backslash$LARGE & \LARGE even larger font \\
					$\backslash$huge & \huge huge font \\
					$\backslash$huge & \Huge largest font \\		\end{tabular}
	\end{center}		
\end{table}

\subsection{Justifications}

\subsubsection{Flushleft, Center, and Flushright}

Justifications are another aesthetic tool for a word processor. In general, \LaTeX\ indents the first paragraph\footnote{You may prevent and indentation by using \texttt{$\backslash$noindent}.} on the left margin, but aligned on the right margin. The author may control the alignment, however, with \emph{environments}---this concept will be developed in a later lecture.

For instance, one would follow the following syntax:
\\
\noindent
\texttt{$\backslash$begin\{\textit{argument}\}}
Desired text\ldots
\texttt{$\backslash$end\{\textit{argument}\}}
Where \textit{argument} is either \texttt{center}, \texttt{flushleft}, or \texttt{flushright}. The text between begin and end will function accordingly. Below is an example code and output of centered text.
\begin{verbatim}
\begin{center}
Text between the beginning and end portion of the command will be 
centered. Try it with flushleft and flushright.
\end{center}
\end{verbatim}
\begin{center}
Text between the beginning and end portion of the command will be 
centered. Try it with flushleft and flushright.
\end{center}

\subsubsection{Quote}

\LaTeX\ has a few predefined justification commands, such as quote.\footnote{Similar justifications are covered more extensively in Lecture 3.1.1.} This will add a block indentation on the left, but remain the font size the same. Below is an example.
\begin{verbatim}
\begin{quote}
The quote environment is best utilized when quoting text that is 
three lines or more. I have used it to make examples stand out.
\end{quote}
\end{verbatim}
\begin{quote}
The quote environment is best utilized when quoting text that is 
three lines or more. I have used it to make examples stand out.
\end{quote}

\subsection{Bullets and Enumerations}

Setting up bulleted or numbered lists are similar to justifications insofar as they are placed between \texttt{$\backslash$begin\{\textit{argument}\}} and \texttt{$\backslash$end\{\textit{argument}\}}. However, when creating a bulleted list, the \textit{argument} is \textt{itemize}. Each item in the list begins with \texttt{$\backslash$item}. You may customize how a bullet looks by placing a character between square brackets after \texttt{item}, such as \texttt{item[-]}.
\begin{verbatim}
\begin{itemize}
\item First item.
\item[-] Second item.
\item[+] Then some more\ldots
\end{itemize}
\end{verbatim}
\begin{itemize}
\item First item.
\item[-] Second item.
\item[+] Then some more\ldots
\end{itemize}

A similar syntax is used for numbered lists. However, a numerated list is \texttt{enumerated}. You can create sub-numbering (or sub-bullets) by creating another list under an item. The following is a large example.
\begin{verbatim}
\begin{enumerate}
\item First item.
\item Second item.
	\begin{enumerate}
	\item Sub-item.
	\item Another sub-item.
	\end{enumerate}
\item Then some more\ldots
	\begin{itemize}
	\item Bulleted item.
	\end{itemize}
\end{enumerate}
\begin{verbatim}
\item First item.
\item Second item.
	\begin{enumerate}
	\item Sub-item.
	\item Another sub-item.
	\end{enumerate}
\item Then some more\ldots
	\begin{itemize}
	\item Bulleted item.
	\end{itemize}
\end{verbatim}

%================================Conclusion

\section{Conclusion}

A lot of material was covered in this section. Table \ref{tbl:logo} lists some basic logos that \LaTeX\ has built-in. Likewise, a number of characters that are used by \LaTeX\ can usually be shown in a document by preceding it with a backslash ($\backslash$)---one notable exception is the backslash which is shown with \texttt{\$$\backslash$backslash\$}. Table \ref{tbl:format} list some basic formatting commands (e.g. italics, bolding). Table \ref{tbl:font} shows the font commands to change the font from the default. 

Bulleted and enumerated lists were also introduced. One notable concept was placing text between \texttt{$\backslash$begin} and \texttt{$\backslash$end}. This form of a command will be brought up again in Lecture 3.1 and is very powerful.

\end{document}


%READ BEFORE EDITING=======================DO NOT DELETE==========================
%Author(s): Tom Schenk Jr.
%Last Revision: 10/25/2005
%Copyright: Licensed under Creative Commons Attribution-NonCommercial-ShareAlike 2.5 License
%Copyright URI: http://creativecommons.org/licenses/by-nc-sa/2.5/
%You may edit and distribute this document for non-commercial services with attributions to previous authors. You may add your name to the \author{} command if you contributed to this document.

\documentclass[pdf]{prosper}
%================================Packages
\usepackage{hyperref}
\usepackage{mflogo}
\usepackage{fancybox}
\usepackage{multicol}
%================================Document Info./Slide 1
\title{Introduction}
\subtitle{How \LaTeX\ Works}
\author{Tom Schenk Jr.}		%Add your name here if you modified this document. (e.g. \author{name1, name2})
\institution{Drake University}
%================================Document Body
\begin{document}
\maketitle
%================================Slide 2
\begin{slide}{This Presentation}
	\begin{itemize}
		\item The process of making a \LaTeX\ document.
		\item Files that you will encounter.
		\item \LaTeX\ editors.
	\end{itemize}
\end{slide}
%================================Slide 3
\begin{slide}{Making a Document}
	\begin{itemize}
		\item Making a document in a Word Processor (i.e. Word, OpenOffice, etc.) usually consists of one file that you use for editing and viewing.
		\item Making a document in \LaTeX\ consists of two main files.
			\begin{itemize}
				\item A ``source file'' that the author edits (.tex).
				\item A viewable output file meant for viewing and distribution (PDF, PostScript (.ps), DVI(.dvi)).
			\end{itemize}
		\item Almost any text editor or word processor can be a \LaTeX\ editor, which can be seamlessly converted to an output file.
	\end{itemize}
\end{slide}
%================================Slide 4
\begin{slide}{Making a Document}
	\begin{itemize}
		\item The (.tex) file is edited and ``compiled'' to make a readable output.
		\item Below is the list of file formats: \\
\begin{tabular}{|c|}
\hline
Source File \\
(.tex) \\
\hline \\
DeVice Independent File \\
(.dvi) \\
\hline \\
Readable Output \\
(PDF, PostScript) \\
\hline
\end{tabular}
		\item Compiling the source file into .dvi and PostScript/PDF is rather easy. (Lecture 2.3)
	\end{itemize}
\end{slide}
%================================Slide 5
\begin{slide}{Making a Document}
	\begin{itemize}
		\item Below is a basic layout of a \LaTeX\ source file:
			\small
				\texttt{$\backslash$documentclass[pdf]\{article\}} \\
				\texttt{$\backslash$usepackage\{hhref\}} \\
				\texttt{$\backslash$title\{The Dogma of Cultural Relativity\}} \\
				\texttt{$\backslash$author\{Scott Belcher\}} \\
				\texttt{$\backslash$begin\{document\}} \\
				\texttt{$\backslash$maketitle} \\
				\texttt{$\backslash$section\{Introduction\}} \\
				\rmfamily\textit{Introductory Material\ldots} \\
				\texttt{$\backslash$section\{Body\}} \\
				\rmfamily\textit{The body of the document\ldots} \\
				\texttt{$\backslash$end\{document\}} \\
	\end{itemize}
\end{slide}
%================================Slide 6
\begin{slide}{Making a Document}
	\begin{itemize}
		\item When you compile the document, \LaTeX\ feeds the document through line-by-line.
		\item Likewise, commands in \LaTeX\ take the following format: \texttt{$\backslash$command\{\textit{argument}\}}.
		\item As the document is fed through, \LaTeX\ recognizes $\backslash$ is the beginning of the command and \{ is the beginning of the argument.
		\item As a consequence, we will have to get use to some small quarks for making paragraphs and new lines. (Lecture 2.2.2)
		\item Likewise, we will have to use some commands to format text (i.e. \textit{italics},\textbf{bolding}) (Lecture 2.2.2)
	\end{itemize}
\end{slide}
%================================Slide 7
\begin{slide}{Files You May Encounter}
	\begin{itemize}
		\item Editing a \LaTeX\ document means editing the .tex document.
		\item There are three output formats: DeVice Independent (.dvi), PostScript (.ps), and Portable Document Format (.pdf).
		\item Other than these four file types, there are auxiliary files that are either created while compiling a document or needed for more advanced features (e.g. bibliographies.
		\item The next slide explicates some file formats you may encounter.
	\end{itemize}
\end{slide}
%================================Slide 8
\begin{slide}{Files You May Encounter}
	\begin{tabular}{l l}
			\hline
			\multicolumn{1}{c}{Extension}							&	\multicolumn{1}{c}{Description} \\
			\hline
			.tex	&	The \LaTeX\ input file. \\
			.log			&	A log of messages from compiling a .tex file. \\
			.bib		&	Contains all of our bibliography information. \\
			.aux		& Contains formatting setting for the bibliography. \\
			.blg		& A log of messages from compiling a .bib file. \\
			.dvi		&	An output file viewable with a special viewer. \\
			.ps			& PostScript format viewable with a viewer \\
							& (e.g. GhostViewer) \\
			.pdf		& Portable Document Format (Acrobat) \\
			\hline
	\end{tabular}
\end{slide}
%================================Slide 9
\begin{slide}{Editors}
	\begin{itemize}
		\item \LaTeX\ input files (.tex) are ASCII files.
		\item This means they are highly portable and can be edited on almost any text editor. Some editors are easier than others.
		\item Generic editors are able to open any document and save using ASCII.
			\begin{itemize}
				\item Notepad, Wordpad for Windows and TextEditor for MacOS.
			\end{itemize}
		\item Specific editors are geared toward \LaTeX\ with syntax highlighting: TeXnicCenter for Windows and AlphaX for MacOS.
		\item We will discuss different editors in more detail later in the tutorial (Lecture 1.3).
	\end{itemize}
\end{slide}
%================================Slide 10
\begin{slide}{Editors}
	\begin{itemize}
		\item Word Processors sometimes have trouble saving in another format (i.e. Works$\rightarrow$Word).
		\item \LaTeX\ is namely concerned about two things: ASCII format and correct syntax.
		\item This means making the same document on different operating systems is fairly easy.
		\item In some ways, \LaTeX\ is meant to be device independent.
	\end{itemize}
\end{slide}
%================================Slide 11
\begin{slide}{Review}
	\begin{itemize}
		\item A \LaTeX\ document differentiates between an editable input file and a viewable output file.
		\item .tex file is the file that is edited.
		\item Basically any text editor can make a \LaTeX\ document.
		\item It is very easily to switch between PC/Mac/Linux/Unix/Solaris while making a document.
		\item NEXT: Installing \LaTeX\
	\end{itemize}
\end{slide}
\end{document}
%READ BEFORE EDITING=======================DO NOT DELETE==========================
%Author(s): Tom Schenk Jr.
%Last Revision: 9/19/2005
%Copyright: Licensed under Creative Commons Attribution-NonCommercial-ShareAlike 2.5 License
%Copyright URI: http://creativecommons.org/licenses/by-nc-sa/2.5/
%You may edit and distribute this document for non-commercial services with attributions to previous authors. You may add your name to the \author{} command if you contributed to this document.

\documentclass[pdf]{prosper}
%================================Packages
\usepackage{hyperref}
\usepackage{mflogo}
\usepackage{fancybox}
\usepackage{multicol}
%================================Document Info./Slide 1
\title{Indexing and Bibliographies}
\subtitle{Bibliographies}
\author{Tom Schenk Jr.}		%Add your name here if you modified this document. (e.g. \author{name1, name2})
\institution{Drake University}
%================================Document Body
\begin{document}
\maketitle
%================================Slide 2
\begin{slide}{This Presentation}
	\begin{itemize}
		\item The bibliography file.
		\item Creating the .bib file.
		\item Inserting references in our document.
		\item Necessary commands (\texttt{$\backslash$bibliographystyle},\texttt{$\backslash$bibliography}).
		\item Other bibliography styles.
	\end{itemize}
\end{slide}
%================================Slide 2
\begin{slide}{Overview of Bibliographies}
	\begin{itemize}
		\item Similar to table of contents, \LaTeX\ is easily able to generate a bibliography at the end of our document.
		\item There are a handful of ways a bibliography can be generated: one method is using \texttt{$\backslash$thebibliography} environment.
		\item We will be using BibTeX to generate our bibliographies---this is more automatic and is generally more appropriate for undergraduate students.
		\item \textit{Not So Short Introduction} covers \texttt{thebibliography} method.
	\end{itemize}
\end{slide}
%================================Slide 2
\begin{slide}{Overview of Bib\TeX}
	\begin{itemize}
		\item The Bib\TeX\ is a program and file structure created by Leslie Lamport (the creator of \LaTeX\mbox{}) and Oren Patashnik.
		\item Using Bib\TeX\ typically involves a command in the preamble defining the bibliography style, a command in the body where we want the bibliography inserted, and another file with a .bib extension with our bibliography information.
		\item A publications title, author(s), year of publication, publisher, etc., divided into fields in the .bib file.
		\item Depending on our bibliography style, \LaTeX\ organizes the order of the fields into a recognizable bibliography style and displays it in our document.
	\end{itemize}
\end{slide}
%================================Slide 2
\begin{slide}{The Bibliography File}
	\begin{itemize}
		\item The most efficient use of Bib\TeX is creating our bibliography file as we research and write our paper.
		\item Authors that we don't reference simply will not be included in the bibliography.
		\item Bibliography files end in .bib, we can use our normal \LaTeX\ editor to create the bibliography file.
		\item No matter which bibliography style we use (e.g. MLA), the .bib file will look the same.
		\item The bibliography style is chosen in our .tex document.
	\end{itemize}
\end{slide}
%================================Slide 2
\begin{slide}{The .bib File}
	\begin{itemize}
		\item First, each bibliography entry will be defined by type (e.g. book, article, manual)---this will tell \LaTeX\ how to organize the entries.
		\item The type is defined by \texttt{@book}, \texttt{@article}, \texttt{manual} for a book, article, and manual, respectively (later we will list typical entry types).
		\item Next to the entry type, we need to create a short name we will refer to each entry.
			\begin{center}
				\texttt{@article\{Smith81,}
			\end{center}
		\item The name of this article entry is \texttt{Smith81}, which we will use when we reference this document.
		\item The open curly bracket \{ is important as well as the comma---we will close the bracket at the end of the bibliography entry.
	\end{itemize}
\end{slide}
%================================Slide 2
\begin{slide}{The .bib File (con't)}
	\begin{itemize}
		\item After declaring the entry type and its name, we start entering the fields of the entry.
		\item Fields are specific values, for instance, \texttt{author = \{Matt Smith\},} is the entry for the author.
		\item Typically in our .bib file, entries are indented and a linebreak after each field.
		\item The next slide gives an example of a bibliography entry.
	\end{itemize}
\end{slide}
%================================Slide 2
\begin{slide}{Bibliography Example}
	\texttt{@article\{Dixit77,} \\
	\indent \texttt{author = \{Avinash Dixit and Joseph Stiglitz\},} \\
	\indent \texttt{year = \{1977\},} \\
	\indent \texttt{title = \{Monopolistic Competition and Optimum Product Diversity\},} \\
	\indent \texttt{journal = \{American Economic Review\},} \\
	\indent \texttt{volume = \{67\},} \\
	\indent \texttt{number = \{3\},} \\
	\indent \texttt{pages = \{297--308\},} \\
	\texttt{\}}
\end{slide}
%================================Slide 2
\begin{slide}{Bibliography Example (con't)}
	\begin{itemize}
		\item The supplemental document contains an example output of a bibliography, remember the style will differ depending on the style we choose (covered later in this lecture).
		\item However, it is important to note the ending curly bracket \} that ends the bibliography entry.
		\item Also note the comma at the end of each field entry.
		\item Puttiing curly brackets around each field entry is optional, but suggested.
		\item In my experience, it's best to name each entry by the author's last name and the year of publication (i.e \texttt{Dixit77}) to easily remember.
		\item Now we can take a look of how to cite a reference.
	\end{itemize}
\end{slide}
%================================Slide 2
\begin{slide}{Referencing}
	\begin{itemize}
		\item In the body of our document, we will use commands to reference entries by the entry name we determined.
		\item We use the \texttt{$\backslash$cite\{\textit{name}\}} will cite \textit{name} (i.e. where \textit{name} would be \texttt{Dixit77} according to our earlier example).
		\item By default, \LaTeX\ uses a numbering system for bibliographies---each bibliography entry is assigned a number by alphabetical order:
			\begin{center}
				[1] Dixit, Avinash and Joseph Stiglitz. Monopolistic Competition and Optimum Product Diversity. \textit{American Economic Review}, \textbf{67},3:297--308.
			\end{center}
		\item In the document, the source will be cited by this number:
			\begin{center}
				Dixit and Stiglitz [1] created a mathematical version of monopolistic competition\ldots
			\end{center}
		\item This format is sometimes used by journals; however, it is generally unacceptable for homework, we will explore a more useful bibliography format later in this lecture.
	\end{itemize}
\end{slide}
%================================Slide 2
\begin{slide}{Necessary commands}
	\begin{itemize}
		\item Running a bibliography requires at least two commands; \texttt{$\backslash$bibliographystyle\{\textit{style}\}} and \texttt{$\backslash$bibliography\{\textit{bibfile}\}}.
		\item \texttt{$\backslash$bibliographystyle\{\textit{style}\}} can be placed anywhere in the document (preamble or body) as long as its before \texttt{$\backslash$end\{document\}}. 
		\item \texttt{$\backslash$bibliography\{\textit{bibfile}\}} can be placed anywhere before in the body, but this command starts the bibliography, so it should be placed immediately before \texttt{$\backslash$end\{document\}}.
	\end{itemize}
\end{slide}
%================================Slide 2
\begin{slide}{bibliographystyle Command}
	\begin{itemize}
		\item \texttt{$\backslash$bibliographystyle\{\textit{style}\}} tells \LaTeX\ the style we want for our bibliography.
		\item The \texttt{plain} style was shown earlier.
		\item The website contains links to other style examples, but these are unacceptable for classes.
		\item Later in the lecture, we will look at the \texttt{natbib} style.
	\end{itemize}
\end{slide}
%================================Slide 2
\begin{slide}{bibliography Command}
	\begin{itemize}
		\item \texttt{$\backslash$bibliography\{\textit{bibfile}\}} begins the bibliography file.
		\item \textit{bibfile} is the .bib file we created, you do not have to insert .bib in this command.
		\item For example, if our bibliography was named \textit{paperbib.bib} on our computer, then we'd use \texttt{$\backslash$bibliography\{paperbib\}} as our command.
		\item If we have multiple bibliographies, we may separate them using a comma; for instance, if our .bib files were named \textit{paperbib.bib} and \textit{morebib.bib}, then we'd use \texttt{$\backslash$bibliography\{paperbib,morebib\}}.
		\item Even when we start changing our bibliography styles, we do not have to change this command.
	\end{itemize}
\end{slide}
%================================Slide 2
\begin{slide}{More Useful Bibliography Styles}
	\begin{itemize}
		\item We mentioned earlier that the default or \textit{plain} bibliography style is unacceptable for academia.
		\item We will introduce the \texttt{natbib} style, which is modeled on the MLA style.
		\item The website contains a link to a complete list of other acceptable styles. %reference reed
	\end{itemize}
\end{slide}
%================================Slide 2
\begin{slide}{More Useful Bibliography Styles}
	\begin{itemize}
		\item To include the \texttt{natbib} style, we must use the \texttt{natbib} package: \texttt{$\backslash$usepackage\{natbib\}}.
		\item With \texttt{natbib} loaded, we need to load the \texttt{plainnat} style using \texttt{$\backslash$bibliographystyle\{\textit{plainnat}\}}.
	\end{itemize}
\end{slide}
%================================Slide 2
\begin{slide}{natbib: Citation Commands}
	\begin{itemize}
		\item \texttt{natbib} has several \textit{cite} commands for different outputs (we will use the \texttt{Dixit77} example from above (more on the website):
			\begin{center}
				\begin{tabular}{l|l}
					\hline \\
					Command & Output \\
					\hline \\
					\texttt{$\backslash$citet\{Dixit77\}} & Dixit and Stiglitz (1977) \\
					\texttt{$\backslash$citep\{Dixit77\}} & (Dixit and Stiglitz, 1977) \\
					\texttt{$\backslash$citealt\{Dixit77\}} & Dixit and Stiglitz 1977 \\
					\texttt{$\backslash$citeauthor\{Dixit77\}} & Dixit and Stiglitz \\
					\texttt{$\backslash$citeyear\{Dixit77\}} & 1977 \\
					\texttt{$\backslash$citeyearpar\{Dixit77\}} & (1977) \\
				\end{tabular}
			\end{center}
		\item In general, \texttt{$\backslash$citet} does not put parenthesis around the name, while \texttt{$\backslash$citep} will
		\item Where there are three or more authors, et al. will be used.
	\end{itemize}
\end{slide}	
%================================Slide 2
\begin{slide}{Generating Bibliographies}
	\begin{itemize}
		\item To generate the Bib\TeX file, \texttt{bibtex} is used.
		\item Similar to a table of contents, we need to repeat a few commands to make sure \LaTeX\ has the correct reference names (we will assume our document is named \textit{capstone}:
			\begin{center}
				\begin{tabular}{l}
					\texttt{latex capstone} \\
					\texttt{bibtex capstone} \\
					\texttt{latex capstone} \\
					\texttt{latex capstone} \\
				\end{tabular}
			\end{center}
	\end{itemize}
\end{slide}
%================================Slide 2
\begin{slide}{Bibliography Review}
	\begin{itemize}
		\item We assign values of the bibliography entry in the documents .bib file.
		\item \texttt{$\backslash$cite\{\textit{name}\} will cite the bibliography entry.}
		\item We need to commands for the bibliography: \texttt{$\backslash$bibliographystyle\{\textit{style}\}} is the bibliography style and \texttt{$\backslash$bibliography} inserts the bibliography.
		\item The \texttt{natbib} style is preferred for homework.
		\item Look at the lecture notes for detailed examples, also look at the supplemental documents for a listing of bibliography styles (especially Reed University's website).
	\end{itemize}
\end{slide}

\end{document}
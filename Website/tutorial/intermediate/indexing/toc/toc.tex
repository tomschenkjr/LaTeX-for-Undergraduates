%READ BEFORE EDITING=======================DO NOT DELETE==========================
%Author(s): Tom Schenk Jr.
%Last Revision: 9/18/2005
%Copyright: Licensed under Creative Commons Attribution-NonCommercial-ShareAlike 2.5 License
%Copyright URI: http://creativecommons.org/licenses/by-nc-sa/2.5/
%You may edit and distribute this document for non-commercial services with attributions to previous authors. You may add your name to the \author{} command if you contributed to this document.

\documentclass[pdf]{prosper}
%================================Packages
\usepackage{hyperref}
\usepackage{mflogo}
\usepackage{fancybox}
\usepackage{multicol}
%================================Document Info./Slide 1
\title{Indexing and Bibliographies}
\subtitle{Bibliographies}
\author{Tom Schenk Jr.}		%Add your name here if you modified this document. (e.g. \author{name1, name2})
\institution{Drake University}
%================================Document Body
\begin{document}
\maketitle
%================================Slide 2
\begin{slide}{This Presentation}
	\begin{itemize}
		\item Different bibliography commands.
		\item Necessary packages.
		\item Preamble commands.
		\item \textit{bibliography} command.
		\item The \textit{.bib} file.
		\item Bibliography styles.
	\end{itemize}
\end{slide}
%================================Slide 2
\begin{slide}{Overview of Bibliographies}
	\begin{itemize}
		\item Similar to table of contents, \LaTeX\ is easily able to generate a bibliography at the end of our document.
		\item There are a handful of ways a bibliography can be generated: one method is using \texttt{$\backslash$thebibliography} environment.
		\item We will be using BibTeX to generate our bibliographies---this is more automatic and is generally more appropriate for undergraduate students.
		\item \textit{Not So Short Introduction} covers \texttt{thebibliography} method.
	\end{itemize}
\end{slide}
%================================Slide 2
\begin{slide}{Overview of BibTex}
	\begin{itemize}
		\item The BibTex is a program and file structure created by Leslie Lamport (the creator of \LaTeX\) and Oren Patashnik.
		\item Using BibTeX typically involves a command in the preamble defining the bibliography style, a command in the body where we want the bibliography inserted, and another file with a .bib extension with our bibliography information.
		\item A publications title, author(s), year of publication, publisher, etc., divided into fields in the .bib file.
		\item Depending on our bibliography style, \LaTeX\ organizes the order of the fields into a recognizable biblography style and displays it in our document.
	\end{itemize}
\end{slide}
%================================Slide 2
\begin{slide}{Packages}
	\texttt{$\backslash$begin\{document\}} \\
	\texttt{$\backslash$tableofcontents} \\
	\texttt{$\backslash$section\{Introduction\}} \\
	\ldots
\end{slide}
%================================Slide 2
\begin{slide}{Table of Contents Example (con't)}
	\begin{itemize}
		\item Look at the supplemental documents at examples of a table for contents.
		\item The table of contents will contain the number, name, and page number of each chapter/section.
		\item The \LaTeX\ document must be generated twice in order to correctly number all the sections.
		
	\end{itemize}
\end{slide}
%================================Slide 2
\begin{slide}{Table of Contents Review}
	\begin{itemize}
		\item Our chapters, sections, and subsections commands can be made into a table of contents at the beginning of the documents.
		\item \texttt{$\backslash$tableofcontents} will insert a table of contents where the command is placed.
		\item The table of contents will contain the number, name, and page number of each chapter/section.
	\end{itemize}
\end{slide}

\end{document}
